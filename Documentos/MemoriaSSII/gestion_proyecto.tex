\chapter{Gesti\'on del proyecto}
\section{Mantenimiento del c\'odigo}

Debido a la cantidad de c\'odigo que se iba a generar, y a los problemas de integraci\'on que pudieran surgir, se decidi\'o usar un sistema de mantenimiento de c\'odigo llamado \emph{CVS}. el CVS nos permit\'\i a trabajar desde nuestras casas y mantemer actualizadas las \'ultimas versiones de los ficheros fuente. Debido a que el CVS nos permite almacenar no s\'olo c\'odigo, lo hemos empleado tambi\'en para guardar cualquier otro dato de inter\'es para el proyecto, tales como documentos, actas de reuniones, ejemplos\dots

La estructura de directorios en el CVS es la siguiente:
\begin {itemize}
\item [Codigo] Aqu\'\i ~se encuentran los ficheros fuente de los distintos lenguajes.
\item [Documentos] Se usa para guardar cualquier tipo de documento de inter\'es, tales como actas, manuales\dots
\item [Ejemplos] Donde guardamos los ejemplos que consideramos mas representativos del momento.
\item [PatronesRitmicos] Donde se encuentran los patrones r\'\i tmicos creados para ser usados por GENARO.
\item [PatronesBateria] Donde se encuentran los patrones de bater\'\i a que iba a usar GENARo.
\item [ProgramasAux] En el que se han guardado otros programas que se usan con GENARO
\item [Timidity] Con el timidity configurado para ser invocado por GENARO.
\end {itemize}

El servidor CVS nos ha sido proporcionado por Berlios, http:\\developer.berlios.de

\section{Reuniones}

Con el fin de poner en com\'un los avances semanales, as\'\i ~como de hacer una peque\~na evaluaci\'on de la situaci\'on en cada momento de GENARO, se fijaron reuniones semanales que ven\'\i an definidas en la \emph{Gesti\'on de la configuraci\'on}, documento que se encuentra en su correspondiente directorio del CVS. De cada una de estas reuniones hay un acta, con el nombre de la fecha en la que tuvo lugar la reuni\'on. Este acta era escrita y pasada a limpio por un secretario, puesto que se cambiaba semanalmente entre los miembros del equipo.

Tambi\'en se realizaron diversas reuniones con el director del proyecto, se intentaban realizar cada 2 semanas, pero en algunas ocasiones, debido a que los avances no eran los esperados, estas reuniones se retrasaban.

\section{Comunicaci\'on interna}

Para mantener el contacto entre los distintos miembros del grupo, se creo una lista de correo electr\'onico para poder difundir con la mayor brevedad posible una noticia a todos los integrantes. Esta lista nos la ofrece Berlios, tras registrar all\'\i ~ nuestro proyecto.

En ocasiones extraordinarias la comunicaci\'on se realizaba mediante llamadas telef\'onicas, a los numeros que vienen en la \emph{Gesti\'on de la configuraci\'on}.

\section{Organizaci\'on de trabajo}

Se ha intentado asignar el trabajo individual de la manera m\'as eficiente posible, para que cada miembro se distribuya su tiempo como mejor considere. Esta asignaci\'on se realizaba en las reuniones semanales.
En ciertas ocasiones, el trabajo ten\'\i a que ser realizado de manera colectiva, en casos como enlazar distintas partes del proyecto. Para ello nos reun\'\i amos en un mismo local todos los integrantes para codificar el c\'odigo necesario.
