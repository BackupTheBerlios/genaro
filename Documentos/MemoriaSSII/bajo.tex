\chapter{Generador de l\'ineas de bajo}
\section{Introducci'on}
Este m'odulo se encarga de para componer \emph{lineas de bajo} a partir de progresiones de acordes. Una linea de bajo es una composici'on hecha para un bajo, no hay diferencia entre una linea de bajo y una melod'ia hecha para una parte de bajo, es solamente cuesti'on de terminolog'ia. Pero se emplea el t'ermino linea de bajo porque tiene ciertas connotaciones: de una linea de bajo se espera que sea r'itmica y repetitiva, circular,que vaya m'as pegada al acorde, siguiendo al acorde de forma m'as subordinada que una melod'ia; y que se coordine bien con la bater'ia, que se \emph{empaste} bien con 'esta. Es decir, se espera que desempe~ne funciones r'itmicas y arm'onicas. De una melod'ia en cambio se espera m'as linealidad, mayor libertad r'itmica y mayor independencia de las notas estables del acorde.\newline
Al igual que con las melod'ias, se emplean las reglas de la armon'ia para que el la linea de bajo resultado este relacionada con la progresi'on y `quede bien` al sonar a la vez que la traducci'on a m'usica de 'esta. Y tambi'en como las melod'ias, las lineas de bajo producidas por Genaro son monof'onicas.\newline
Hay 3 bajistas correspondientes a 3 algoritmos de generaci'on de bajo: 
        \begin{enumerate}
        \item El bajista \emph{Fundamentalista} es casi determinista, y acompaña a cada acorde tocando su nota fundamental (la más estable del acorde). Solamente hay aleatoriadad en la elecci'on de la octava empleada para cada acorde.
        \item El bajista \emph{Aphex} compone aplicando mutaciones aleatorias similares a las de la melodía, sobre un bajo compuesto por el fundamentalista.
        \item El bajista \emph{Walking} interpola las notas del bajista fundamentalista haciendo que las notas intermedias tengan una duraci'on especificada, y luego muta el resultado de forma similar a Aphex.
        \end{enumerate}
Cada bajista se dise~n'o extiendo al anterior, aumentando el nivel de complejidad del algor'itmo.
	
\section{Abstracciones empleadas}
Este m'odulo fu'e de los 'ultimos en hacerse y por ello se beneficia de las lecciones aprendidas durante el desarrollo de los dem'as m'odulos. En cambio, por ser uno de los 'ultimos m'odulos es que se hizo con m'as prisa y quiz'as por ello no se desarrollaron abstracciones adicionales para 'este m'odulo. O quiz'as es que tampoco son muy necesarias, visto que las mutaciones aplicadas a listas de m'usicas funcionan bien para simular variaciones de una melod'ia. Es un tema sobre el que habr'ia que pensar con calma en las ampliaciones futuras de genaro, en todo caso la versi'on actual del generador de lineas de bajo de Genaro utiliza:
        \begin{enumerate}
        \item Curvas mel'odicas: solamente el bajista Walking las emplea, gener'andolas de una manera muy concreta que se explicar'a en la secci'on de implementaci'on.
        \item Progresiones de acordes: todos los bajistas hacen lo mismo en realidad, tocar las fundamentales de los acordes y meter m'as o menos notas entre medias.
        \item No utiliza listas de acentos: El ritmo se produce de distintas maneras seg'un el bajista:
                \begin{enumerate} 
                \item Fundamentalista: lo marca la progresi'on y nada m'as
                \item Aphex: lo marca la progresi'on y las mutaciones, que dividen el tiempo en potencias de dos, algo apropiado en el contexto binario en que trabaja Genaro. La aplicaci'on de un n'umero suficiente de mutaciones produce variaciones r'imticas muy ricas.
                \item Walking: determinado fundamentalmente por la duraci'on que se le pasa de par'ametro, las mutaciones pueden tambi'en a~nadir variaciones r'itmicas interesantes.
                \end{enumerate}
        \end{enumerate}

Este m'odulo, como el de la melod'ia, tambi'en utiliza el concepto de \emph{registro}, que recordemos, es una restricci'on sobre las alturas de las notas, que deben pertenecer a un intervalo determinado por 'este. El bajo tiene asociado un registro grave, como es l'ogico.


\section{Algoritmo de generaci'on de l\'ineas de bajo}
Como en el caso de la generaci'on de melod'ias, lo dif'icil es generar la m'usica para un s'olo acorde, despu'es generar la m'usica para toda la progresi'on consiste simplemente en enlazar la m'usica para cada acorde. La idea fundamental en la que se basan los algor'itmos es que, recordemos, \emph{todos los bajistas hacen lo mismo en realidad, tocar las fundamentales de los acordes y meter m'as o menos notas entre medias}. Teniendo eso en cuenta, veamos por separado los algoritmos para cada tipo de bajista:
	\begin{enumerate}
	\item Aphex: Este algoritmo recibe como entrada dos alturas correspondientes a la fundamental del acorde para el que hace la linea, y al acorde siguiente. Por tanto son las alturas entre las que le bajista debe insertar notas. Tambi'en recibe  el tipo de escala correspondiente al acorde, y la duraci'on de 'este, a la que se deber'a ajustar la l'inea. Por 'ultimo recibe una serie de par'ametros enteros. El algortimo funciona como sigue:
		\begin{enumerate}
		\item[(i)] Primero construye una lista de dos notas compuesta por una nota de altura igual a la primera suministrada, y de durac'on igual a la suministrada, seguida de otra nota de altura igual a la segunda suministrada y de una duraci'on cualquiera (se pone siempre la misma pero no importa). Las listas de notas son la entrada a los algoritmos mutadores, as'i que con esta lista, que es una primera versi'on de la linea de bajo, ya tenemos algo que mutar.
		\item[(ii)] Se muta esta primera linea de bajo con las mutaciones que introducen notas intermedias, que son las mismas mutaciones \emph{dividir notas} y \emph{dividir notas fino} que emple'abamos para la melod'ia. El n'umero de veces que se aplican y el par'ametro de la segunda mutaci'on vienen dados por los enteros entrada del algoritmo. Y se aplican de forma entrelazada (no se aplica primero $n_{1}$ veces una mutaci'on y luego $n_{2}$ veces la otra, sino que cada vez se elige aleatoriamente cu'al de las dos aplicar, dando la misma probabilidad a las dos, hasta que se hayan aplicado el n'umero de veces especificado por los par'ametros.
		\item[(iii)] Se elimina la 'ultima nota de la linea mutada. Esto se hace porque esa 'ultima nota era la correspondiente a la fundamental del acorde siguiente, y en realidad no pertenece a esta linea sino a la del acorde siguiente. Adem'as la duraci'on de esa nota se puso sin seguir ning'un criterio, porque se sab'ia desde un principio que iba a terminar elimin'andose. Sin embargo era necesario poner esa nota porque las mutaciones de notas intermedias necesitan tener dos alturas entre las que poner notas.
		\item[(iv)] Se muta esta nueva linea con una variaci'on de la mutaci'on `alargar notas` de la melod'ia llamada \emph{alargar notas destructivo}. Esta mutaci'on es similar, pero considera candidato a alargarse cualquier nota no silencio que sea seguida por cualquier nota, sea silencio o no. Despu'es elige uno solo de los candidatos, dando probabilidad mayor de ser elegido a los que corresponden a alturas m'as estables dentro del acorde. Desp'ues alarga la nota elegida eleminando la nota que la sigue y sumando la duraci'on de 'esta a la de la nota elegida. Se aplica tantas veces como indica el par'ametro de entrada correspondiente.
		\end{enumerate}
	\item Fundamentalista: El bajo fundamentalista en realidad no tiene un algoritmo propio, sino que es un caso degenrado del bajista walking en el que el n'umero de veces que se aplican las mutaciones es 0.
	\item Walking: Este algoritmo recibe como entrada la misma informaci'on que el bajista Aphex, a~nadiendo tambi'en la duraci'on de las notas de la l'inea, y el desplazamiento m'aximo que se emplear'a para generar la curva. Estos dos par'ametros se explicar'an mejor cuando aparezcan en el algoritmo, que funciona como sigue:
		\begin{enumerate}
		\item[(i)] Se calcula la distancia en la escala entre las dos alturas de entrada, entre las que se quiere generar la l'inea. Con esta distancia se construye la curva mel'odica que tiene esta distancia como 'unico punto. Esta curva representa el salto desde la primera altura a la segunda. 
		\item[(ii)] El par'ametro que indica la duraci'on de las notas de la linea indica la duraci'on que tendr'an casi todas las notas de la l'inea, antes de mutar la l'inea con la mutaciones usadas en el bajista Aphex. Decimos casi todas porque es posible que la duraci'on a la que hay que ajustar la l'inea no sea divisible por la duraci'on de las notas y haya que a~nadir una sola nota de otra duraci'on para ajustarla. Teniendo esto en cuenta se calcula cuantas notas de esa duraci'on (m'as quiz'as una adicional) necesitamos para ocupar la duraci'on del acorde.
		\item[(iii)] Aplicamos a la curva mel'odica inicial una mutaci'on especial que desplaza un punto de la curva elegido al azar, una cantidad aleator'ia entre -(desplazamiento m'aximo) y +(desplazamiento m'aximo); y que despu'es inserta en un posici'on de la curva elegida al azar otro punto de valor contrario al desplazamiento que se realiz'o. De esta manera se consigue otra curva con un punto m'as y que realiza un desplazamiento total (suma de todos los puntos de la curva) igual al de la curva de partida. Aplicamos esta mutaci'on a la curva inicial un n'umero de veces igual al n'umero de notas calculadas en el paso (ii) menos uno, para obtener una curva mel'odica con n'umero de puntos igual a ese n'umero de notas.
		\item[(iv)] A esta curva mel'odica se le quita el 'ultimo elemento y se le añade de primer elemento un 0, para que al aplicarla sobre la altura del acorde la primera altura que aparezca sea dicha altura.
		\item[(v)] Construimos una lista de duraciones para las notas de la l'inea. Ser'a una lista cuyo 'unico elemento sea la duraci'on del acorde si 'este dura menos que la duraci'on especificada para las notas. Si no, si la duraci'on del acorde es divisible por la de las notas ser'a una lista en la que todos los elementos son la duraci'on de las notas. Si no es divisible ser'a una lista en la que todos los elementos menos uno son la duraci'on de las notas, y en la que el elemento distinguido ser'a la duraci'on que se an~ade para encajar la linea a la duraci'on del acorde. La posici'on de ese elemento distinguido se elegir'ia al azar.
 		\item[(vi)] Se mezclan las dos listas generadas en los puntos (iv) y (V) para conseguir una primera versi'on de la l'inea de bajo.
		\item[(vii)] Esta primera versi'on de la l'inea de bajo se muta tantas veces como indiquen los par'ametros, aplicando las mutaciones siempre en este orden: alargar notas destructivo,  dividir notas y dividir notas fino.
		\end{enumerate}
	\end{enumerate}
Una vez sabemos generar una l'inea de bajo para una acorde generarla para una progresi'on es sencilla. Se distribuyen las mutaciones entre los acordes seg'un sus duraciones y se llama al algoritmo de generaci'on de lineas para acordes correspondiente al bajista especificado. Las lineas siempre van de un acorde al siguiente excepto para el 'ultimo acorde que enlaza con el primero.

\section{Implementaci'on}
Se apoya en la librer'ia Haskore de Haskell, y est'a desarrollada 'integramente en este lenguaje. Tambi'en se apoya mucho en el funciones desarrolladas para el m'odulo de melod'ias.
\newline\newline
El 'unico tipo nuevo definido fu'e el tipo de bajista:
	\begin{verbatim}
data TipoBajista = Aphex | Walking | Fundamentalista
     deriving(Enum,Read,Show,Eq,Ord,Bounded)
	\end{verbatim}

\section {Otros usos de 'este m'odulo}
Al escuchar las m'usicas resultado resulta sorprendente observar la capacidad mel'odica del bajista Aphex, que en ocasiones es mejor melodista que el m'odulo de melod'ia. En un futuro quiz'as podr'ia ofrecerse como otra opci'on de generaci'on de melod'ias.
\newline
La combinaci'on de varios bajistas Walking a distintas velocidades (es decir, para distintas duraciones de nota) da como resultado m'usicas que recuerdan a las polifon'ias barrocas. Podr'ia ser un punto de partida para empezar un m'odulo con esta direcci'on.


\section {Capacidad de ampliaci'on}
Como siempre, la ausencia de objetos e interfaces y las abstracciones definidas favorecen la ampliabilidad de este m'odulo, proporcionando puntos de acceso para la generaci'on de melod'ias. Las vias m'as evidentes por las que se podr'ia ampliar este m'odulo son:
\begin{enumerate}
	\item Adici'on del registro como par'ametro de la generaci'on de lineas, para poder generar l'ineas m'as agudas y emplearlas como melod'ias.
	\item Uso de cromatismos en el bajo: utilizar alturas muy inestables, que no pertenecen a la escala, como \emph{notas de paso} en las l'ineas. Esto dotar'ia de m'as tensi'on y moviento a la m'usica.
	\item Algo totalmente diferente: Cualquier programa en cualquier lenguaje que lea archivos de texto en el formato especificado por el predicado es\_progresion/1, y ficheros de patr'on r'itmico puede sustituir a este m'odulo.
	\end{enumerate}
