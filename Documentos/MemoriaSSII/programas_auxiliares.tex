\chapter{Herramientas auxiliares}
\section{Timidity}
Timidity es un sintetizador software, realizado por Tuukka Toivonen, Masanao Izumo. Lo hemos usado por ser GNU.

\section{swi prolog}
Interprete y compilador de prolog, tambi\'en es GNU. Su opci\'on de compilar archivos prolog a ejecutables ha sido de gran ayuda para el intercambio de informaci\'on.

\section{Lilypond}
Es un compilador de un lenguaje de descripci\'on de partituras a .pdf o formato similar. Tambi\'en es GNU.

\section{C++ builder}
Es un compilador de C++ que posee facilidades para generar una GUI.

\section{Hugs 98}
Es un interprete de Haskell. Tiene una licencia que permite la libre distribuci\'on.

\section{Haskore}
Haskore son unas librer\'\i as de haskell que brindan facilidades para el tratamiento musical. Se fundamenta en el tipo \emph{Music}, que es un tipo recursivo que representa la m\'usica. Tambi\'en proporciona funciones para generar archivos midi a partir de elementos de tipo Music, y viceversa. El tipo Music est\'a definido de la siguiente manera:
\begin{verbatim}
data Music = Note Pitch Dur [NoteAttribute]   -- a note \ atomic 
           | Rest Dur                         -- a rest /    objects
           | Music :+: Music                  -- sequential composition
           | Music :=: Music                  -- parallel composition
           | Tempo  (Ratio Int) Music         -- scale the tempo
           | Trans  Int Music                 -- transposition
           | Instr  IName Music               -- instrument label
           | Player PName Music               -- player label
           | Phrase [PhraseAttribute] Music   -- phrase attributes
    deriving (Show, Eq)
type Dur   = Ratio Int                        -- in whole notes
type IName = String
type PName = String
\end{verbatim}

