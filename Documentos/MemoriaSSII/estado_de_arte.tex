\chapter{Estado del arte en la composici\'on aleatoria de m\'usica}
\section{Introducci\'on}

En esta secci\'on haremos un peque\~no recorrido por los sistemas que hay para la creaci\'on de m\'usica, y la manera en la que trabajan.

\section{Programas que juegan con la m\'usica}

Existen algunos programas que permiten crear o modificar  piezas. Hablaremos de algunos de ellos para conocer un poco el campo de la creaci\'on de m\'usica antes de llegar nosotros.

\subsection{Juego de los dados de Mozart}

Uno de los primeros sistemas de composici\'on autom\'atica que conocemos, fue el juego que ide\'o el m\'usico \emph{Wolfgang Amadeus Mozart}. Con este juego puedes componer cientos de minuetos, seleccionando distintos fragmentos musicales de unas tablas mediante el resultado de una pareja de dados.

\subsection{JAMMER}

JAMMER es un programa que permite generar, mediante par\'ametros especificados, acompa\~namientos musicales o arreglos a una canci\'on. JAMMER no crea una canci\'on de la nada, trabaja con una pieza que altera para obtener una nueva.
Puedes conocer m\'as sobre JAMMER en http://www.soundtrek.com.

\subsection {KeyKit}

KeyKit es un lenguaje de programaci\'on musical creado por Tim Thompson. Es la base para algunos programas de composici\'on de m\'usica que enumeraremos a continuaci\'on:

\begin {itemize}

\item [Muse-O-Matic] El usuario ha de introducir una palabra, y obtiene como resultado una canci\'on que se ha creado basandose en esa palabra. Este algoritmo es determinista, para una palabra generar\'a siempre la misma canci\'on.
\item [Web Tones] Genera una pieza musical a partir de una p\'agina web. 
\item [Key Chain] Permite elegir al usuario entre distintas transformaciones que aplica a la canci\'on.
\item [Pieces-O-MIDI] Toma un midi y lo descompone en piezas, que mezcla para generar un nuevo midi
\item [GIF Jam] Toma una imagen en formato gif, y crea un fichero midi dependiendo de los colores de cada pixel
\item [Expresso] Este programa usa algoritmos fractales para conseguir funciones bastantes complejas en terminos de una variable X, una nota sustituye la variable X y se eval\'ua la expersi\'on.
\item [Fresh Roast] Es uun refinamiento de \emph{Expresso}, que incluye bater\'\i as.
\item [Life Forms] Se basa en el \emph{Juego de la vida} para crear la canci\'on.

\end {itemize}