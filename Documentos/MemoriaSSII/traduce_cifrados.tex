\chapter{Traducci\'on de cifrados}

\section{Introducci\'on}
La finalidad del m\'odulo de traducci\'on de cifrados hecho en Haskell es que proporcione todas las notas
exactas que poseer\'a cada cifrado en funci\'on, por supuesto, de ciertas opciones de 
entrada y, en caso de ambig\"uedad, usara la aleatoriedad para elegir entre posibles resultados.\\
\indent Cuando nos referimos a las notas exactas de un cifrado musical nos referimos a las
alturas de las notas, tanto el \emph{PitchClass} como la octava, es decir, aquello que determina la 
frecuencia del sonido. No nos referimos a c\'omo esas notas se van a ejecutar en el tiempo 
porque eso es parte de la siguiente etapa en la que vamos a encajar estas notas con lo que hemos
llamado patron r\'\i tmico. Por ello no es necesario tener en cuenta la duraci\'on individual de 
cada nota, que ser\'\i a lo \'unico que nos faltar\'\i a para tener una verdadera nota musical, 
sino que basta tener la duraci\'on del acorde. Es por ello que la salida de este m\'odulo es el tipo: \\

\small
\begin{verbatim}
type AcordeOrdenado = ([Pitch],Dur)
\end{verbatim}
\normalsize

es decir, la duraci\'on del acorde junto con las notas que tiene.\\
\indent Pongamos un ejemplo. El cifrado de C Maj7 (do mayor con s\'eptima mayor), posee las notas
(C,E,G,B) pero esa informacion esta incompleta para una traduccion verdadera. Supongamos
que buscamos que ese acorde tenga 7 notas, est\'e en estado fundamental (el C como nota m\'as
grave) y en primera disposici\'on (el C como nota m\'as aguda). Es obvio que tenemos que 
duplicar algunas alturas para que se cumpla lo anterior. El programa actuar\'a devolviendo lo
siguiente: (C,E,G,B,C,G,C). Ahora s\'\i~que se cumple lo anterior pero la traducci\'on sigue 
siendo incompleta seg\'un lo dicho anteriormente ya que falta la octava. Digamos que ahora queremos que la nota
m\'as grave est\'e en la cuarta octava y las dem\'as notas sean la siguiente menor m\'as aguda. Entonces el acorde quedar\'\i a como sigue:
((C,4),(E,4),(G,4),(B,4),(C,5),(G,5),(C,6)) que s\'\i ~es lo que buscamos.\\
\indent La etapa de traducci\'on se compone de dos fases:

\begin{enumerate}
\item Traducci\'on a forma fundamental: simplemente indica los \emph{PitchClass} que posee
el acorde
\item Reparto de las notas en voces: duplica y permuta los \emph{PitchClass} de la etapa 1
para que se ajusten a unos determinados par\'ametros. Vamos adelantando que los par\'ametros
m\'as importantes son los dos sistemas de traducci\'on: el continuo y el paralelo.
\end{enumerate}

\section{Traducci\'on a forma fundamental}

\subsection{Vectores de matr\'\i culas}

Un cifrado se compone de dos partes: el grado sobre el que se asienta y la especie
que es. Nosotros hemos llamado a la especie \emph{matr\'\i cula}, de esa forma, por ejemplo, 
el cifrado de \emph{C Maj7} posee de matr\'\i cula \emph{Maj7} y de vector de \emph{PitchClass} [C,E,G,B]; el cifrado
de \emph{C m} (do menor) posee matricula \emph{m} y \emph{PitchClass} [C,bE,G].\\
\indent Independientemente del grado del cifrado todos los acordes que son mayores posee
una tercera mayor y una quinta justa (un m\'usico entender\'\i a esto perfectamente aunque
no es necesario para lo que nos trata a continuaci\'on). Una tercera mayor est\'a compuesta
de 4 semitonos (\'o 2 tonos) y una quinta justa de 7 semitonos (\'o 3 tonos y medio). De esa
forma es f\'acil ver que para obtener un acorde mayor sobre cualquier nota simplemente
tenemos que sumar 4 semitonos a dicha nota (obteniendo la tercera del acorde) y de nuevo
7 (para obtener la quinta). M\'as f\'acil es ver todav\'\i a que eso se puede expresar en forma
de tupla como (0,4,7). Veamos un ejemplo para aclararlo:\\
\indent Buscamos el acorde de D mayor. El \emph{PitchClass} D en \emph{Haskore} se representa por el n\'umero 2, 
cosa que nos facilita la suma. Si sumamos 2 + (0,4,7) nos da (2,6,9) que pas\'andolo otra vez
a notas nos da (D,Fs,A) (Fs es Fa sostenido). Aqui hay que decir que tanto \emph{Haskore} como MIDI usa un sistema
atemperado, es decir, que las notas Fs y Gf (Sol bemol) representan el mismo sonido.
Una ver entendido este concepto pasamos a mostrar todos los vectores asociados con
las matr\'\i culas que maneja Genaro:\\
Mayor $\to$ [0,4,7]\\
Menor $\to$ [0,3,7]\\
Au $\to$ [0,4,8]\\
Dis $\to$ [0,3,6]\\
Sexta $\to$ [0,4,7,9]\\
Men6 $\to$ [0,3,7,9]\\
Men7B5 $\to$ [0,3,6,10]\\
Maj7 $\to$ [0,4,7,11]\\
Sept $\to$ [0,4,7,10]\\
Men7 $\to$ [0,3,7,10]\\
MenMaj7 $\to$ [0,3,7,11]\\
Au7 $\to$ [0,4,8,10]\\
Dis7 $\to$ [0,3,6,9]\\
\indent Las funciones de Haskell que encapsulan esto se encuentran en el modulo \emph{Progresiones.hs}
y se llaman \emph{matriculaAVector} y \emph{sumaVector}.


\subsection{Traducci\'on a forma fundamental}
Una vez que tenemos a nuestra disposici\'on los vectores de las matr\'\i culas ya podemos
traducir cualquier cifrado a su forma fundamental:

\small
\begin{verbatim}
traduceCifrado :: Cifrado -> [PitchClass]
traduceCifrado (grado, matricula) = map  absPitchAPitchClass 
        (sumaVector (matriculaAVector matricula) (gradoAInt grado))
\end{verbatim}
\normalsize

\section{Reparto de las voces}
Existen dos formas principales de ordenas las notas de la etapa anterior: sistema paralelo
y sistema continuo. Ambas formas han sido inspiradas del libro de Enric Herrera.\\

\subsection{Sistema Paralelo}
El sistema paralelo es el m\'as simple. B\'asicamente se encarga de colocar todos los acordes
en una inversi\'on y una disposicion dada. Los parametros que necesitan que sean pasados 
por el usuario son los siguientes:

\begin{enumerate}
\item octava inicial: en la octava a partir de la cual se empiezan a colocar las notas\\
\item inversi\'on: un entero que indica la inversi\'on que es\\
\item disposici\'on: otro entero que indica la disposici\'on del acorde\\
\item n\'umero de notas: n\'umero de notas final del acorde\\
\end{enumerate}

La inversi\'on de un acorde es simplemente la indicaci\'on de qu\'e nota es la m\'as grave en un acorde.
Una inversion de 1 indica que el acorde esta en primera inversi\'on (la tercera del acorde 
como nota m\'as grave) y as\'\i . La inversi\'on 0 se usa para decir que estamos en estado fundamental
(la fundamental del acorde como nota m\'as baja).\\
\indent Algo parecido pasa con la disposici\'on que \'unicamente depende de la nota m\'as aguda.
Una disposici\'on de 1 indica que la fundamental del acorde es la nota m\'as alta y as\'\i .\\
\indent Tanto la inversi\'on como la disposicion depende \'unicamente de la nota
m\'as grave y de la m\'as aguda y eso implica dos cosas. Primera es que el n\'umero m\'inimo de
notas que puede tener un acorde es dos (las dos notas extremas). Se podria haber hecho de
otra forma pero esta es la que vimos m\'as coherente con los par\'ametros. Segundo es que
no hay informaci\'on sobre la colocaci\'on del resto de las notas. Podriamos haber seguido
una opci\'on parecida a la del sistema continuo (como ya veremos) pero pensamos que iba
m\'as encaminado con la filosofia de este sistema (seg\'un Enric Herrera, por supuesto)
que las notas intermedias se colocaran en el mismo orden que establece su estado fundamental.\\
\indent Una nueva abstracci\'on tenemos que introducir en este momento: el AcordeSimple.\\

\small
\begin{verbatim}
type AcordeSimple = [Int] 
\end{verbatim}
\normalsize

Un acorde simple es solamente una lista de n\'umeros. Cada n\'umero es un \'indice sobre la lista de
notas del acorde que se pasa de la etapa anterior, es decir, sobre un acorde en estado 
fundamental. Para formar un acorde simple seg\'un los parametros del sistema paralelo usamos\\

\small
\begin{verbatim}
formarAcordeSimple :: NumNotasFund -> NumNotasTotal -> Inversion 
                      -> Disposicion -> AcordeSimple
formarAcordeSimple numNotasFund numNotasTotal inv disp = 
        reverse (disp : anadeAcordeSimple numNotasFund (numNotasTotal - 2) [inv + 1])
\end{verbatim}
\normalsize

�Par\'a qu\'e nos sirve esto? Simplemente independiza las notas verdaderas del acorde con su
ordenaci\'on. Esto se entiendo muy f\'acilmente con un ejemplo:\\
Supongamos el acorde de C Mayor de nuevo cuyo estado fundamental es [C,E,G] que es proporcionado
por la etapa 1. Queremos este acorde con 10 notas, en primera inversi\'on, y en tercera
disposici\'on. Llamando a la funci\'on \emph{formarAcordeSimple} tenemos:

\small
\begin{verbatim}
TraduceCifrados> formarAcordeSimple 3 10 1 3 
[2,3,1,2,3,1,2,3,1,3] 
\end{verbatim}
\normalsize

�Qu\'e significa esto? Significa que el primer elemento es el segundo elemento de [C,E,G],
el segundo es el tercer elemento de [C,E,G] y as\'\i . La importancia de esto es que lo
que hemos llamado acorde simple es independiente de las notas concretas a las que se refiere.\\
\indent Ahora simplemente hay que fusionar las dos cosas pero eso es muy f\'acil ya que son \'indices
(recordemos que con el 1 nosotros nos referimos al primer elemento, a diferencia de en
Haskell que se refiere al primer elemento con el 0)

\small
\begin{verbatim}
encajaAcordeSimple :: [PitchClass] -> AcordeSimple -> [PitchClass]
encajaAcordeSimple lpc [] = []
encajaAcordeSimple lpc (num : resto) = (lpc !! (num - 1)) : encajaAcordeSimple lpc resto
\end{verbatim}
\normalsize

Ahora s\'olo nos queda la parte de a~nadir la octava.

\subsection{Sistema Continuo}
La idea principal de este sistema es que las voces de un acorde se muevan lo menos
posible respecto a las voces del acorde anterior, de esa forma el efecto transitorio entre
acordes ser\'a m\'as suave.\\
\indent M\'as detalladamente se busca lo siguiente

\begin{enumerate}
\item si un acorde posee notas en com\'un respecto del anterior entonces esas notas tienen que 
ser la misma
\item el resto de las voces se mueve a la nota m\'as cercana.
\end{enumerate}

Enric Herrera propone en la p\'agina 42 un m\'etodo que nos hace saber cuantas notas en com\'un 
poseen esos dos acordes. En primer lugar ese m\'etodo es s\'olo para triadas aunque se podr\'\i a
ampliar sin mucha dificultad a cuatriadas. Por otro lado s\'olo tiene en cuenta los acordes
diat\'onicos. Eso es un problema si queremos usar algo parecido sobre acordes no diatonicos
como puede ser los dominantes secundarios. Por tanto este m\'etodo no es suficientemente 
general para nuestros propositos y tenemos que ampliarlo.\\
\indent El algoritmo se divide en cuatro partes.

\begin{enumerate}
\item Traducimos todos los cifrados a su estado funtamental sin duplicar ni permutar los PitchClass, es decir, lo que optenemos
directamente de usar el Vector de la matricula.
\item Permutamos cada acorde hasta que la diferencia de las notas sea la m\'as peque�a con respecto al acorde anterior. 
\item Duplicamos las notas para que posean el n\'umero de notas deseadas.
\item Insertamos la octava para que la altura de las notas este fijada definitivamente.
\end{enumerate}

Ahora vamos a verlos en detalle.

\subsubsection{Traducci\'on a estado fundamental}
La traducci\'on a estado fundamental no merece demasiada atenci\'on ya que se realiza de la misma
forma a la del sistema paralelo, usando la funci\'on \emph{traduceCifrado}.

\subsubsection{Obtenci\'on del acorde m\'as cercano al anterior}
Esta es, seg\'un mi criterio, la parte m\'as importante del algoritmo del sistema continuo.
La base del algoritmo es bastante f\'acil. En primer lugar necesitamos dos operaciones
sobre listas de PitchClass: una que nos diga las notas en com\'un en la misma posici\'on
de la lista y otra que nos diga la distancia entre dos listas de PitchClass. Esto nos da
lo siguiente:

\small
\begin{verbatim}
coincidencias :: [PitchClass] -> [PitchClass] -> Int
coincidencias lp1 lp2 = foldr1 (+) (map fromEnum [(lp1 !! i) == 
                        (lp2 !! i) | i <- [0..(longMen - 1 )]])
	where longMen = min (length lp1) (length lp2)

distancia :: [PitchClass] -> [PitchClass] -> Int
distancia lp1 lp2 = foldr1 (+) [abs ( ((map pitchClass lp1) !! i) - 
                    ((map pitchClass lp2) !! i) ) | i <- [0..(longMen - 1 )] ]
	where longMen = min (length lp1) (length lp2)
\end{verbatim}
\normalsize

Una vez que tenemos estas dos funciones ya podemos sacar el acorde m\'as cercano a otro de 
la siguiente forma. Primero hacemos todas las permutaciones de la lista de PitchClass que
representa el acorde en posici\'on fundamental. Eso no es tan preocupante computacionalmente 
hablando ya que nosotros vamos a tener como m\'aximo cuatriadas que forman listas de cuatro
elementos (4�` combinaciones). Segundo vamos a sacar los acordes cuya funci\'on \emph{coincidencias}
nos de el mayor valor y eliminar los otros de la lista. Tercero vamos a sacar los acordes (de los que quedan)
cuya funci\'on distancia al anterior sea la menor. Cuarto, de los restantes vamos a elegir uno al azar.
Muy posiblemente ya en la segunda etapa solo nos quede una lista de un elemento aunque
hay que asegurarse hasta el final.\\
\indent El primer acorde de la progresi\'on es un caso especial ya que no tiene ning\'un acorde anterior con
el que poder compararse. Lo que hacemos es simplemente elegir una inversi\'on aleatoria
para \'el.\\
\indent El algoritmo completo se recoge en la funci\'on organiza que es: \\

\small
\begin{verbatim}
organizar :: RandomGen a => a -> [[PitchClass]] -> [[PitchClass]]
organizar gen (x:xs) = elegido : organizarRec sigGen elegido xs
    where lista = [inversion i x | i <- [0..(length x - 1 )]];
          (elegido, sigGen) = elementoAleatorio gen lista

organizarRec :: RandomGen a => a -> [PitchClass] -> [[PitchClass]] -> [[PitchClass]]
organizarRec _ _ [] = []
organizarRec gen referencia (x:xs) = nuevaRef : organizarRec sigGen nuevaRef xs
    where (nuevaRef, sigGen) = masCoincidente gen referencia x
\end{verbatim} 
\normalsize

donde inversion es la i-esima inversion de la lista x.

\subsubsection{Duplicacion de notas}
La idea es la de usar el llamado AcordeSimple para acceder a las notas ya ordenadas de los acordes. Ahora todos
los acordes simples se van a forman empezando desde el n\'umero 1. Eso no implica que el acorde vaya a estar en forma
fundamental ya que el \'indice 1 sobre la lista de PitchClass no corresponde a la fundamental del acorde porque la
etapa anterior se ha encargado de permutar la lista adecuadamente. El c\'odigo de \'este algoritmo se encuentra
integrado en al misma funci\'on que la del siguiente por lo que se ver\'a despu\'es c\'omo es.

\subsubsection{Insercion de octava}
Ahora solamente falta insertar la octava en las listas de PitchClass para que tengan la altura correspondiente.
Suponemos que el acorde anterior ya tiene todas las octavas a~nadidas y buscamos introducirlas al actual.
No podemos hacer que empiecen a a�adirse desde la m\'as grave como se hizo en el sistema paralelo porque eso podria
darnos alturas diferentes para notas que en teor\'\i a son iguales. Hay que hacer lo siguiente: 
buscar la primera nota coincidente entre los dos acordes y tomar esa octava, 
dividir el acorde actual por dos por dicha nota. Una vez dividido la parte m\'as grave
del acorde se le asigna octavas de forma descendente y la m\'as aguda de hace de forma creciente. Como las notas comunes
est\'an en la misma posici\'on dentro de las listas de PitchClass nos asegura que dichas notas comunes reciben la misma octava

\small
\begin{verbatim}
traduceProgresionSistemaContinuo :: RandomGen a => a -> OctaveIni 
        -> NumNotasTotal -> Progresion -> [AcordeOrdenado]
traduceProgresionSistemaContinuo gen octaveIni numNotasTotal progresion 
	= zip (arreglaTodos octaveIni (map2 encajaAcordeSimple 
            (organizar gen (map traduceCifrado cifrados)) listaAcordesSimples)) 
            duraciones
		where desabrochar = unzip progresion ;
			cifrados = fst desabrochar ;
			duraciones = snd desabrochar ;
			listaAcordesSimples = map reverse [anadeAcordeSimple (matriculaAInt 
                                        ((map snd cifrados)!!i)) (numNotasTotal-1) [1] 
                                        | i <- [0..(length cifrados-1)] ]


\end{verbatim}
\normalsize

donde \emph{arreglaTodos} se encarga de introducir la octava por el procedimiento ahora dicho
y \emph{organiza} traduce los cifrados por el sistema continuo.

\section{Mejoras}
Pocas objeciones tengo respecto de este m\'odulo. Quiz\'as lo que menos me ha gustado
ha sido la forma en que se introduce la octava inicial. En el sistema paralelo
no veo que tenga m\'as remedio pero en el sistema continuo habr\'\i a sido bueno implementar
un m\'etodo para que las notas de los acordes se mantuviera entre ciertos limites definidos por el usuario. El m\'etodo
actual puede hacer que las notas de los acordes subas o bajen m\'as y m\'as de un acorde
al siguiente llegando incluso a pasarse de las octavas permiticas si introducimos gran cantidad
de acordes. Por suerte esto no pasar\'a debido a la funci\'on \emph{eliminaOctavasErroneas}
que deja las octava menores de 0 a 0 y las mayores de 21 a 21 (0 y 21 son los l\'\i mites del
estandar MIDI). No es lo mejor pero, por lo menos, no da error.








