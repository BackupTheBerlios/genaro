\chapter{Bater\'\i a}

\section{Introducci\'on}
La parcusi\'on es la parte instrumental que va a marcar el ritmo. Bebe de las ideas
de los \emph{patrones r\'\i tmicos} y de los \emph{acordes ordenados}, por tanto, si esas partes
se han entendido no va a costar ning\'un trabajo entender \'esta.

\section{Los instrumentos de la bater\'\i a de Genaro}
Los intrumentos que va a tener Genaro en su bater\'\i a son los siguientes:
\small
\begin{verbatim}
data PercusionGenaro = Bombo | CajaFuerte | CajaSuave | CharlesPisado | CharlesAbierto
                       | CharlesCerrado | TimbalAgudo | TimbalGrave | Ride | Crash
     deriving (Show, Read, Eq, Ord, Enum, Ix)
\end{verbatim}
\normalsize
que son guardados en la lista
\small
\begin{verbatim}
bateriaGenaro :: [PercusionGenaro]
bateriaGenaro = [ toEnum i | i <- [0..9] ]
\end{verbatim}
\normalsize

\section{Implementaci\'on}

\subsection{Percusi\'on en Haskore}
El Haskore de Hudak posee su propia definici\'on de tipos para la percusion que
se encuentra en el tipo \emph{PercussionSound}. Dicha asociaci\'on es una biyecci\'on
con los intrumentos de percusi\'on que posee MIDI. En el estandar MIDI cada
instrumento de percusi\'on esta asociado a una altura (\emph{Pitch}, es decir, tipo de nota y octava)
que es tocado por el canal 10.\\
\indent Nosotros vamos a usar parte de esos instrumentos pero le vamos a dar otro nombre por
comodidad. Por consiguiente tenemos que hacer una funci\'on de asocie bidireccionalmente
ambas nomenglaturas.

\small
\begin{verbatim}
bateriaHaskore :: [PercussionSound]
bateriaHaskore = map de_PercusionGenaro_a_PercusionHaskore bateriaGenaro

asociacionPercusion :: [(PercusionGenaro, PercussionSound)]
asociacionPercusion = [(Bombo, BassDrum1),
                       (CajaFuerte, LowMidTom ),
                       (CajaSuave, HiMidTom ),
                       (CharlesPisado, PedalHiHat ),
                       (CharlesAbierto, OpenHiHat ),
                       (CharlesCerrado, ClosedHiHat ),
                       (TimbalAgudo, LowTimbale ),
                       (TimbalGrave, HighTimbale ),
                       (Ride, RideCymbal1 ),
                       (Crash, CrashCymbal1 )]



de_PercusionGenaro_a_PercusionHaskore :: PercusionGenaro -> PercussionSound
de_PercusionGenaro_a_PercusionHaskore pg = ph
      where (lpg, lph) = unzip asociacionPercusion;
            Just ind = elemIndex pg lpg;
            ph = lph !! ind


de_PercusionHaskore_a_PercusionHaskore :: PercussionSound -> PercusionGenaro
de_PercusionHaskore_a_PercusionHaskore ph 
      | isJust quizas_ind    = lpg !! (fromJust quizas_ind)
      | isNothing quizas_ind = error ("de_persucionHaskore_a_PercusionHaskore " 
            ++ show ph ++ ":No se puede traducir a PercusionGenaro")
      where (lpg, lph) = unzip asociacionPercusion;
            quizas_ind = elemIndex ph lph
\end{verbatim}
\normalsize

\subsection{Usando conocimientos anteriores}
Vamos a usar aquello que tenemos hecho de patrones r\'\i tmicos y de acordes ordenas
para formar la bateria.\\
\indent Como ya hemos dicho antes cada instrumento de percusi\'on de MIDI es representado
por una altura (pitch). La uni\'on de todos ellos en una lista
junto con una duraci\'on obtendriamos un acorde ordenado:

\small
\begin{verbatim}
acordeOrdenadoBateria :: Dur -> AcordeOrdenado
acordeOrdenadoBateria d = (map f bateriaHaskore,d)
       where f = pitch.(+35).fromEnum

acordeOrdenadoBateria' :: Dur -> [AcordeOrdenado]
acordeOrdenadoBateria' d = [acordeOrdenadoBateria d]
\end{verbatim}
\normalsize

Si ahora le aplicamos a ese acorde ordenado un patr\'on r\'\i tmico vamos a repartir
dichas notas en el tiempo.

\small
\begin{verbatim}
encajaBateria :: Dur -> PatronRitmico -> Music
encajaBateria d pr = deAcordesOrdenadosAMusica Ciclico (Truncar1, Truncar2) 
                     pr (acordeOrdenadoBateria' d)
\end{verbatim}
\normalsize

\indent El patr\'on r\'\i tmico pasado, aunque puede ser cualquiera, debe ser uno que tenga 
10 voces de altura, es decir, tantas como instrumentos de percusion. Es por
ello que los patrones de encaje verticales nos son indiferentes ya que encaja
perfectamente.\\
\indent Se podria haber hecho que aceptara cualquier patr\'on pero las diferentes formas
de encaje no tienen aqui tanto sentido como cuando el acorde ordenado representa
notas de un instrumento. Por tanto se descarto esa posibilidad, no aqui, en Haskell,
sino en el interfaz gr\'afico.\\
\indent Por ultimo s\'olo queda que el \emph{Music} de salida sea ejecutado por el instrumento
''Drums'' de Haskore para que sea enviado al canal 10 cuando se pase a MIDI.
Pero esta parte no se hace en el m\'odulo \emph{Bateria.hs} sino en el \emph{main.hs} que es
el m\'odulo con el que se comunica el interfaz.\\
\indent Veamos un ejemplo:

\small
\begin{verbatim}
bateriaMusic :: Music
bateriaMusic = Instr "Drums" (encajaBateria (2%1) patronRitmicoCualquiera)
\end{verbatim}
\normalsize

\section{Resultado final}
Al final Genaro no tiene bateria por falta de tiempo. Quiz\'as un tiempo
de dedicaci\'on de una hora o dos hubiera bastado para tenera termina.

\section{Posibles mejoras}
Como habr\'a observado el avispado lector la bater\'\i a de Genaro no tiene
nada de aleatoriedad. Hubiera sido interesante meter algo de ello. Las ideas
que se me ocurrieron a m\'i~con ayuda de companeros de clase son las siguientes:

\begin{enumerate}
\item Que las partes fuertes de un comp\'s sean tocadas por un instrumento o varios
(por ejemplo el bombo) y las d\'ebiles por otro. Una vez hecho esto se podr\'\i a
mutar ligeramente para que se note continuidad a la vez que cambio.\\ 
Una mutacion que se me ocurre es cambiar un instrumento por otro. Por ejemplo, 
si el bombo toca la parte fuerte hacer que pase a tocarla el crash.
\item Hacer que la bater\'\i a se adapte a la duraci\'on de un acorde. Ya que los
acordes no tienen la misma duraci\'on podr\'\i amos hacer, por ejemplo, que el primer
tiempo fuerte sea tocado por el bombo y el resto por una caja.
\item Hacer que la bater\'\i a se adapte a una progresi\'on de acordes. Una t\'ecnica 
de grupos modernos es intercambiar los intrumentos que tocan en la parte
fuerte y debil en los \'ultimos compases de la progresi\'on. Por ejemplo, 
si una caja se golpea en tiempo fuerte que 
pase golpearse en tiempo d\'ebil. Ello da sensaci\'on de que algo va a cambiar.
\end{enumerate}
