\documentclass[a4paper]{report}
\usepackage[spanish]{babel}
\usepackage{musixtex}
\title{Patrones Ritmicos}
\author{Roberto Torres de Alba}

\setlength{\hoffset}{-2cm}
\setlength{\voffset}{-2cm}
\setlength{\textwidth}{450pt}

\begin{document}

\tableofcontents

\chapter{Patrones R\'\i tmicos}

\section{Introducci\'on}
La etapa del patr\'on r\'\i tmico es la que se ejecuta posteriormente a la de traducci\'on de los cifrados. Es por ello que
lo que recibe esta etapa es una lista de lo que hemos llamado ''acordes ordenados'' ( el tipo emph{[AcordeOrdenado]} ). La idea principal 
de esta parte es repartir las voces de los acordes ordenados en el tiempo. Es por ello que vamos a buscar una estructura
que sea capaz de organizar las voces en el tiempo, que sirva para cualquier acorde independientemente del n\'umero de voces
o duraci\'on que posea y qu se pueda guardar para posteriores usos.\\
\indent Para ilustar estas ideas veamos un ejemplo simple. Supongamos que de la etapa anterior de traducci\'on de cifrados nos
llega lo siguiente [ ([(C,4),(E,4),(G,4),(C,4)],1%1) ]. Con estas notas podemos hacer, entre otras cosas, tocarlas todas
a la vez o arpegiarlas:\\

\begin{music}
\instrumentnumber{1}
\setname1{}
\setstaffs11
\startextract
\notes\ibu0f0\zw c\zw e\zw g\wh j\enotes
\bar
\notes\ibu0f0\qa c\qa e\qa g\qa j\enotes
\endextract
\end{music}

Aunque las notas son las mismas ellas estan dispersas en el tiempo de diferente forma.\\

\section{Tipo y Funcionamiento}
El tipo del patron ritmico es el siguiente:

\small
\begin{verbatim}
type Voz = Int      -- Voz: la primera voz, en este caso, es la mas grave
type Acento = Float -- Acento: intensidad con que se ejecuta 
                    --         esa nota. Valor de 0 a 100
type Ligado = Bool  -- Ligado: indica si una voz de una columna esta 
                    --         ligada con la se la siguiente columna.
type URV = [(Voz, Acento, Ligado)]	-- Unidad de ritmo vertical, especifica todas 
                                    -- las filas de una unica columna

type URH = Dur 				-- Unidad de ritmo horizontal, especifica 
                                    -- la duracion de una columna

type UPR = ( URV , URH)             -- Unidad del patron ritmico, especifica completamente 
                                    -- toda la informacion necesaria para una columna 
                                    -- (con todas sus filas) en la matriz que representa el patron

type AlturaPatron = Int			-- numero maximo de voces que posee el patron

type MatrizRitmica = [UPR]          -- Una lista de columnas, vamos, como una matriz

type PatronRitmico = (AlturaPatron, MatrizRitmica)
\end{verbatim}
\normalsize

La idea m\'as simple de ver un patr\'on r\'\i tmico es como una matriz (aunque no lo sea exactamente ya que el n\'umero de filas
por cada columna var\'\i a) que representa una plantilla con agujeros. Su altura (n\'umero de filas) es el n\'umero de 
voces y la anchura (n\'umero de columnas) una duraci\'on en forma de fracci\'on de Haskore. Dicha plantilla se encaja encima 
del acorde ordenado; aquellas voces que caigan dentro de un agujero se ejecutar\'an en ese instante de tiempo. Por ejemplo, supongamos la
siguiente plantilla con agujeros que representa el arpegio anterio de cuatro voces:

\begin{verbatim}
- - - X
- - X -
- X - -
X - - -
\end{verbatim}

donde las X representan los agujeros. Ahora supongamos que la plantilla es esta:

\begin{verbatim}
- X - -
- - X -
- - X -
X - - X
\end{verbatim}

entonces para el siguiente acorde tendriamos lo siguiente cuando aplic\'aramos el patron:

\begin{music}
\instrumentnumber{1}
\setname1{}
\setstaffs11
\startextract
\notes\ibu0f0\zw c\zw e\zw g\wh j\enotes
\bar
\notes\ibu0f0\qa c\qa j\zq e\qa g\qa c\enotes
\endextract
\end{music}

es decir, primero la voz 1 (recordemos que para facilitarme el trabajo ahora las voces se numeran de abajo a arriba), luego 
la voz 4, posteriormente la 2 y la 3 a la vez y, por \'ultimo, la 1 otra vez.\\
\indent Esta forma de ver las cosas es \'util porque el editor de patrones r\'\i tmicos est\'a implementado para que se parezca
a esto aunque el tipo \emph{PatronRitmico} de Haskell necesita m\'as informaci\'on para completarse. Por ejemplo, el \'ultimo patr\'on r\'\i tmico en 
Haskell se escribir\'\i a asi:

\small
\begin{verbatim}
patronEj1 :: PatronRitmico
patronEj1 = ( 4, [( [(1,100,False)] , 1%4),
                  ( [(4,100,False)] , 1%4),
                  ( [(2,100,False),(3,100,False)] , 1%4),
                  ( [(1,100,False)] , 1%4)] )
\end{verbatim}
\normalsize

\section{Encaje del Patron Ritmico}
Ahora vamos a ver como fusionar un patr\'on r\'\i tmico con un acorde ordenado para que se forma la salida definitiva
de esta etapa, que es un tipo Music de Haskore y ya representa m\'usica (e incluso se puede pasar a midi).

\subsection{Encaje perfecto}
Primeramente vamos a ver como encajar un patr\'on cuando el n\'umero de voces, tanto del patr\'on como del acorde ordenado, 
es el mismo y cuando la duraci\'on de ambos tambi\'en es la misma. De esa forma todo encaja perfectamente.\\
\indent El encaje se hace en dos partes. En una primera parte, la que corresponde m\'as claramente con un encaje, se sustituye
cada emph{Voz} dentro del patr\'on r\'\i tmico por un emph{Pitch}.

\small
\begin{verbatim}
encaja :: [Pitch] -> URV -> [(Pitch, Ligado, Acento)]
encaja lp [] = []
encaja lp ( (voz, acento, ligado) : resto) = 
      ( lp !! (voz-1), ligado, acento) : encaja lp resto
\end{verbatim}
\normalsize

Posteriormente se introduce en el Pitch la duraci\'on y el Acento formando una emph{Note} de Haskore.

\small
\begin{verbatim}
insertaDur :: Dur -> [(Pitch, Ligado, Acento)] -> [(Music, Ligado)]
insertaDur dur [] = [(Rest dur, False)]
insertaDur dur lp = insertaDur2 dur lp

insertaDur2 :: Dur -> [(Pitch, Ligado, Acento)] -> [(Music, Ligado)]
insertaDur2 dur [(pitch, ligado, acento)] = [ ( Note pitch dur [Volume acento] , ligado ) ]
insertaDur2 dur ((pitch, ligado, acento) : resto) = 
      ( Note pitch dur [Volume acento] , ligado ) : insertaDur2 dur resto
\end{verbatim}
\normalsize

De esa forma obtenemos una estructura que hemos llamado emph{NotasLigadas}.

\begin{verbatim}
type NotasLigadasVertical = [(Music,Ligado)] 
type NotasLigadas = [(NotasLigadasVertical,Dur)]
-- Donde Dur es la duracion del acorde
\end{verbatim}

La razon que fundamenta el uso de esta estructura es, simplemente, que Haskore no nos
da la opci\'on, dentro de sus constructoras del tipo Music, de hacer que una \emph{Note} se ligue
con la siguiente \emph{Note}. Es por eso que necesitamos un paso previo antes del tipo Music que
se espera que se devuelva en esta etapa. Dicho paso puede ser perfectamente transparente a la
persona que use este m\'odulo.\\
\indent El algoritmo para resolver lo anterior es dif\'\i cil de implementar aunque f\'acil de entender.
Dada la lista que hemos llamado NotasLigadas se comienza por el final de dicha lista y se 
avanza hacia atr\'as. Si la nota actual lleva en la tupla un \emph{True} en el tipo \emph{Ligado} entonces 
se ligara (si se puede) con una de las notas siguientes. Por empezar desde el final de 
la lista cada paso de ligado es correcto y cuando lleguemos a la cabeza la lista completa estara arreglada.\\

\small
\begin{verbatim}
eliminaLigaduras :: NotasLigadas -> NotasLigadas
eliminaLigaduras [n] = [n]
eliminaLigaduras ((notasVerticales, dur) : resto) = 
  let { eliminadas = eliminaLigaduras resto ;
        cabeza = head eliminadas ;
        arregladoCabeza = buscaTodasNotas notasVerticales (fst cabeza)
  } in (fst arregladoCabeza, dur) : (snd arregladoCabeza, snd cabeza) : tail eliminadas

buscaTodasNotas :: NotasLigadasVertical -> NotasLigadasVertical 
                   -> ( NotasLigadasVertical , NotasLigadasVertical )
buscaTodasNotas notas1 notas2 = buscaTodasNotas2 notas1 [] notas2

buscaTodasNotas2 :: NotasLigadasVertical -> NotasLigadasVertical -> NotasLigadasVertical 
                    -> (NotasLigadasVertical, NotasLigadasVertical)
buscaTodasNotas2 [] notas1 notas2 = ( notas1, notas2 )
buscaTodasNotas2 (( nota, False ) : restoPitch ) notas1 notas2 = 
      buscaTodasNotas2 restoPitch ((nota,False):notas1) notas2
buscaTodasNotas2 (( Note pitch dur lA, True ) : restoPitch ) notas1 notas2 = 
	buscaTodasNotas2 restoPitch ((Note pitch (dur + buscaNota pitch notas2) lA, False):notas1)
      (eliminaNota pitch notas2)
\end{vebatim}
\normalsize

Una vez que hemos arreglado todas las notas respecto a sus ligaduras solamente nos queda
pasar la matrix \emph{NotasLigadas} a un tipo Music secuenciando las columnas y paralelizando
las filas.

\small
\begin{verbatim}
-- deNotasLigadasAMusica: dada la lista de notas ligadas (ya sea bien arregladas o no) 
-- las pasa a musica Haskore
deNotasLigadasAMusica :: NotasLigadas -> Music
deNotasLigadasAMusica = deNotasLigadasAMusica2 (0%1)

-- deNotasLigadasAMusica2: es la funcion recursiva de deNotasLigadasAMusica y con acumulador.
-- El parametro dur indica la duracion que hay que dejar hasta el comiento 
-- de la cancion antes de interpretar las notas ligadas a tratar
deNotasLigadasAMusica2 :: Dur -> NotasLigadas -> Music
deNotasLigadasAMusica2 dur [(nV,_)] = Rest dur :+: paraleliza nV
deNotasLigadasAMusica2 dur ((nV,dur2):resto) = (Rest dur :+: paraleliza nV) :=: 
                                               deNotasLigadasAMusica2 (dur + dur2) resto 

-- paraleliza: ejecuta en paralelo a lista de musica sin intereserse por el parametro booleano
paraleliza :: [( Music, Bool )] -> Music
paraleliza [] = Rest (0%1)
paraleliza [ ( nota, _ ) ] = nota
paraleliza (( nota, _ ):resto) = nota :=: paraleliza resto

\end{verbatim}
\normalsize

Como Haskore no posee un Music vac\'\i o usamos \emph{Rest (0\% 1)} para referirnos a algo in\'util 
musicalmente hablando.

\subsection{Problemas de encaje}
En la secci\'on anterior hemos hablado de c\'omo es el encaje del patr\'on cuando el n\'umero
de voces y la duraci\'on de tanto el patr\'on como del acorde ordenado eran los mismos.
Pero en la mayor\'\i a de los casos no va a ser as\'\i . Generalmente el 
acorde ordenado tendr\'a diferente duraci\'on a la del patr\'on y muy posiblemente el n\'umero de 
voces de ambos ser\'a diferente. Ser\'\i a un error crear un nuevo patr\'on para cada acorde
con sus peculariedades as\'\i ~que hemos introducido opciones para solucionar eso.

\subsubsection{Problemas de encaje horizontal}
Se produce cuando la duraci\'on del patr\'on es diferente a la duraci\'on del acorde ordenado
al que se quiere encajar.\\
\indent Si la duraci\'on del patr\'on es menor entonces el problema no es muy importante ya que vamos a repetir
dicho patr\'on tantas veces como haga falta hasta rellenar el acorde completamente o hasta que nos sobre algo, 
en cuyo caso pasaremos a la siguiente opci\'on.\\
\indent Si la duraci\'on del patr\'on es mayor que el del acorde o, dicho de otra forma, sobra
algo del patr\'on cuando lo encajamos en el acorde ordenado entonces la pregunta es
?' qu\'e hacemos con el trozo que sobra ? A nosotros se nos ha ocurrido dos cosas:\\
La primera es que ese trozo se pierda comenzando el siguiente acorde con, otra vez, el 
principio del patr\'on r\'\i tmico.\\
La segunda es que dicho trozo sea el comienzo del siguiente
acorde, es decir, que se le pase al siguiente acorde.\\
La primera opci\'on la hemos llamado encaje horizontal no c\'\i clico y a la segunda encaje horizontal c\'\i clico.
Veamos un ejemplo para asentar ideas:\\

Sea el patr\'on r\'\i tmico del arpegio de cuatro voces que hemos visto antes

\begin{verbatim}
- - - x
- - x -
- x - -
x - - -
\end{verbatim}
Ahora veamos el caso no c\'\i clico
\begin{verbatim}
 - - - x - - - - - - - - - x - - - - - - x - - - x - - 
 - - x - - - x - - x - - x - - - - - - x - - - x - - - 
 - x - - - x - - x - - x - - - - x - x - - - x - - - x 
 x - - - x - - x - - x - - - x x - x - - - x - - - x - 
|-------------|-----|---------|---|-------------------|   duracion acorde
\end{verbatim}
Recordemos que la misma voz en diferentes acordes no tiene porque corresponder a la
misma altura de nota.\\
Ahora veamos el caso c\'\i clico para la misma sucesi\'on de acordes
\begin{verbatim}
 - - - x - - - x - - - x - - - x - - - x - - - x - - - 
 - - x - - - x - - - x - - - x - - - x - - - x - - - x 
 - x - - - x - - - x - - - x - - - x - - - x - - - x - 
 x - - - x - - - x - - - x - - - x - - - x - - - x - - 
|-------------|-----|---------|---|-------------------|   duracion acorde
\end{verbatim}

\subsubsection{problemas de encaje vertical}
Algo parecido ocurre cuando el n\'umero de voces del acorde y el del patr\'on es diferente pero
en este caso tenemos m\'as opciones. Veamos los diferentes casos.\\
Si la altura del acorde es mayor que la del patr\'on:\\
\textbf{1. Truncar:} las voces extra del acorde se ignoran
\begin{verbatim}
         -
 - - - - |
 - - - - |
 - - - x | altura del acorde
 - - x - |
 - x - - |
 x - - - |
         -
\end{verbatim}
\textbf{2. Saturar:} supongamos que el acorde tiene A voces y el patr\'on tiene P. En este caso
las voces desde P+1 hasta A se tocan a la vez cuando el patron toca su voz P-esima.
\begin{verbatim}
         -
 - - - x |
 - - - x |
 - - - x | <- P-esima voz del patron 
 - - x - |
 - x - - |
 x - - - | altura del acorde
         -
\end{verbatim}

Si la altura del acorde es menor que la del patr\'on:\\
\textbf{1. Truncar} (no cunfundir con el truncar anterior ya que son casos distintos aunque
tengan el mismo nombre): Las voces desde A+1 hasta P no se tocan.
\begin{verbatim}
          -
 - x - -  |
 x - - -  | altura del acorde (solo 2 voces)
          -
\end{verbatim}
\textbf{2. Saturar }(no cunfundir con el saturar anterior ya que son casos distintos aunque
tengan el mismo nombre): Las voces desde A+1 hasta P no se tocan se toman como la voz
A-esima del acorde.
\begin{verbatim}
          -
 - x x x  |
 x - - -  | altura del acorde (solo 2 voces)
          -
\end{verbatim}
\textbf{3. C\'\i clico:} En este caso lo que se cambia es el acorde para que tenga el mismo n\'umero
de voces que el patr\'on. Ello se consigue repitiendo las notas del acorde pero aumentando
la octava de forma que el acorde siga ordenado en altura.
\begin{verbatim}
                                              -
 - - - x                             - - - x  |
 - - x -  -     pasaria a ser asi    - - x -  |
 - x - -  |                          - x - -  |
 x - - -  |                          x - - -  | 
          -                                   -
\end{verbatim}
\textbf{4. M\'odulo:} Si una voz X del patr\'on est\'a en el rango A+1 y P, es decir, que cae
fuera del rango del patr\'on, entonces esa voz se transforma en X := ((X-1) mod A) + 1. Eso
nos asegura que X cae entre 1 y A.
A-esima del acorde.
\begin{verbatim}
          -
 - x - x  |
 x - x -  | altura del acorde (solo 2 voces)
          -
\end{verbatim}

Los detalles de implementaci\'on no merecen la pena ponerlos aqu\'\i ~ya que son muy simples. 
Cabe se~nalar que simplemente hay que cambiar lo que nosotros hemos llamado unidad
de ritmo vertical (\emph{URV=[(Voz,Acento,Ligado)]}) a~nadiendo o quitando voces
para que se ajusta a las voces del acorde.

\section{Mejoras}
\subsection{Patrones ritmicos dinamicos}
Lo que hemos dicho hasta aqu\'\i ~podr\'\i a definirse como el uso de patrones r\'\i tmicos est\'aticos, es decir,
que no se generan en tiempo de llamada. Una posible mejora ser\'\i a introducir patrones r\'\i tmicos
din\'amicos, que dado algunos par\'ametros te generara un patr\'on r\\i tmico. Obviamente esto s\'olo se podr\'\i a
hacer para patrones simples y que cumpliesen alguna propiedad escalable. El ejemplo m\'as claro es el del 
arpegio o el de que toca todas las notas a la vez. Para el arpegio din\'amico podr\'\i amos hacer una
funci\'on que, dado una duraci\'on para las notas y un n\'umero de notas, devolviera un patr\'on
r\'\i tmico correspondiente con un arpegio de esas caracter\'\i sticas:

\small
\begin{verbatim}
arpegioDinamico :: Dur -> Int -> PatronRitmico
arpegioDinamico d n = ( n, [ ([(i,100,False)] , d) | i<-[1..n] ] )
\end{verbatim}
\normalsize

Algo tan simple como eso dar\'\i a mucho juego y flexibilidad al programa. 

\subsection{Patrones aleatorios}
Ser\'\i a algo as\'\i ~como un patr\'on din\'amico pero la matriz ser\'\i a generada aleatoriamente. 
Quiz\'as introducir conceptos (posiblemente el ritmo) que restringieran esa aleatoriedad podr\'\i a hacer m\'as audible
un patr\'on de estas caracter\'\i sticas.

\end{document}











