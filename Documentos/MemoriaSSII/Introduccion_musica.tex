\documentclass[a4paper,12pt]{article}
\usepackage[spanish,activeacute]{babel}
%en author tengo que poner los acentos asi pq el activeacute aún no funciona en author
\author{Juan Rodr\'iguez Hortal\'a}
\title{Introducci\'on}
\frenchspacing
\begin{document}
\maketitle
\tableofcontents
\section{?`Qu'e es una nota?}
Para intentar formalizar la m'usica consideremos el m'inimo elemento del que se compone cualquier creaci'on musical, una  nota. Una \emph{nota} es un sonido al que se le han aplicado dos restricciones:
\begin{itemize}
\item De duraci'on: se espera que el sonido que es la nota permanezca sonando durante una cantidad de tiempo determinada. Esta cantidad de tiempo se especifica mediante una \emph{figura}, que no es sino una fracci'on que expresa la duraci'on de la nota en relaci'on con una medida de tiempo absoluta llamada \emph{tempo}, que se especifica para toda la obra musical o para cada fragmento de 'esta.
\item De altura: se espera que el sonido tenga una altura determinada en la escala de medida musical, es decir, que sea m'as o menos grave o agudo. Las \emph{alturas musicales} no son sino frecuencias de oscilaci'on de ondas sonoras, se consideran ciertas frecuencias como v'alidas y entonces se dice que una nota est'a \emph{afinada} cuando corresponde a una de estas frecuencias. Por tanto la altura de una nota es una correspondencia con una de estas frecuencias afinadas. Existen varios sistemas de afinaci'on (justa, temperada, ...) que no son sino correspondencias entre nombres de alturas musicales y frecuencias de oscilaci'on.
\end{itemize}

Por 'ultimo hay que tener en cuenta que en la m'usica la ausencia de sonidos es tan importante como su presencia. Por ello definimos un tipo de notas especiales que son los \emph{silencios}, que son notas con una duraci'on asociada pero sin una frecuencia asociada (o tambi'en podemos considerar que est'an asociadas a una frecuencia especial). Los silencios en una composici'on para un instrumento corresponden con los momentos en que dicho instrumento no emite ning'un sonido.


\section{Dimensiones de la m'usica}
Habiendo entendido lo que es una nota podemos intentar desarrollar una visi'on m'as amplia de la m'usica. Si entendemos las dos restricciones que se imponen sobre las notas como restricciones sobre dos dimensiones sobre las que se desarrolla la musica.
\begin{itemize}
\item Horizontal/Temporal: la m'usica discurre a lo largo del tiempo.
\item Vertical/Altura: los componentes de la m'usica, las notas, tienen unas alturas definidas, es decir, se disponen a lo largo de un eje vertical.
\end{itemize}

Por tanto la m'usica se puede representar como una agrupaci'on de `puntos` (las notas), dispuestas en el espacio seg'un dos ejes: el horizontal o temporal, que determina qu'e sonidos suenan o dejan de sonar, antes o despu'es; y el vertical o de alturas, que determina la altura de los sonidos en el momento de su instanciaci'on.

\section{Otros conceptos b'asicos}
En esta secci\'on introduciremos algunos conceptos que necesitamos manejar para entender el funcionamiento de GENARO

\begin{itemize}

\item [Acorde:] El concepto de \emph{acorde} se puede entender a distintos niveles de abstracci'on. En el nivel m'as concreto, un acorde es un grupo de dos o m'as notas que suenan a la vez. Pero a un nivel m'as alto de abstracci'on un acorde se puede entender como una agrupaci'on de alturas musicales que se considera que tienen un significado conjunto, y llamaremos \emph{cifrado} a la representaci'on de los acordes a este nivel de abstracci'on. 

\item [Tonalidad:] Es un eje o contexto global que define la estabilidad de las notas musicales y la forma en la que se suceden los acordes. GENARO trabaja siempre en tonalidades mayores (en realidad trabaja siempre en Do Mayor y luego traspone el resultado).
 
\item [Comp\'as:] Es una unidad de tiempo en la que se divide una frase u obra musical, cada comp\'as est\'a dividido en periodos de tiempo de igual duraci\'on llamados ''tiempos''. Hay varios tipos de compases, GENARO trabaja siempre en el comp\'as binario de 2/2. Este tipo de comp\'as se caracteriza por ser la sucesi\'on de un tiempo fuerte y un tiempo d\'ebil.

\item [Escala:] Un acorde define una jerarqu\'\i a de sonidos en el intervalo de tiempo en el que este est\'a vigente. Una escala es un conjunto de alturas musicales que se ''permiten'' durante la duraci\'on del acorde. Dentro de las notas de las alturas que pertenecen a la escala, algunas son m\'as estables y otras menos.

\end{itemize}

\section{Enfoque Genaro de la m'usica}
Se puede entender que la m'usica esta formada por tres elementos, \emph{melod'ia}, \emph{armon'ia} y \emph{ritmo}:
\begin{itemize}
\item Melod'ia: se refiere a la sucesi'on de notas a lo largo del tiempo. La melod'ia de una composici'on es lo que la gente suele cantar e identificar m'as claramente.
\item Armon'ia: se refiere a la m'usica considerada seg'un su eje vertical, es decir, a las relaciones de altura entre distintas notas que suenan a la vez. A estos conjuntos de notas simult'aneas los llamamos \emph{acordes}.
\item Ritmo: se refiere a la repetici'on de patrones de duraciones de notas a lo largo del eje temporal.
\end{itemize}

Genaro se inspira en una de las formaciones cl'asicas del Jazz, el tr'io base, a la hora de enfocar la composici'on de m'usica. Un tr'io base est'a compuesto por un piano, un contrabajo y una bater'ia. Estos instrumentos tienen unas funciones o papeles bien definidos, para poder dar cobertura a los tres elementos de los que se compone la m'usica:
\begin{itemize}
\item \emph{Bater'ia}: se ocupa de dar soporte al ritmo.
\item \emph{Bajo}: se ocupa de dar apoyo r'itmico a la bateria y sobre todo de sustentar la armon'ia junto con el piano. Ocasionalmente puede realizar funciones mel'odicas en mayor o menor medida.
\item Piano: el piano se puede entender como dos instrumentos en uno porque cada mano del piano es independiente (si el pianista es bueno). Debido a ello en Genaro consideramos por separado cada mano del piano:
        \begin{enumerate}
        \item[(a)] Mano izquierda: esta mano se ocupa de apoyar al ritmo y sobre todo de la armon'ia. En Genaro llamamos \emph{acompa~namiento} a este `instrumento`.
        \item[(b)] Mano derecha: la mano derecha del piano lleva casi todo el peso mel'odico. En Genaro llamamos \emph{melod'ia} a este `instrumento`.
        \end{enumerate}
\end{itemize}

En Genaro llamamos \emph{tipo de pista} a estos `instrumentos` y la m'usica se genera de forma diferente para cada tipo de pista, adecu'andose a sus caracter'isticas. Al final del proyecto no di'o tiempo a terminar el compositor para pistas de bater'ia, pero si quedaron listos los compositores de acompa~namiento, bajo y melod'ia.

\end{document}
