\documentclass[a4paper,11pt]{article}
\usepackage{latexsym}
\usepackage{graphicx}
\usepackage[spanish]{babel}
%\author{Juan Rodr\'\i guez Hortal\'a, Javier G\'omez Santos, Roberto Torres de Alba}
\title{Herramientas auxiliares}
\frenchspacing
\begin{document}
\maketitle
\tableofcontents
\section{Timidity}
Timidity es un sintetizador software, realizado por Tuukka Toivonen, Masanao Izumo. Lo hemos usado por ser GNU.

\section{swi prolog}
Interprete y compilador de prolog, tambi\'en es GNU. Su opci\'on de compilar archivos prolog a ejecutables ha sido de gran ayuda para el intercambio de informaci\'on.

\section{Lilypond}
Es un compilador de un lenguaje de descripci\'on de partituras a .pdf o formato similar. Tambi\'en es GNU.

\section{C++ builder}
Es un compilador de C++ que posee facilidades para generar una GUI.

\section{Hugs 98}
Es un interprete de Haskell. Tiene una licencia que permite la libre distribuci\'on.

\section{Haskore}
Haskore son unas librer\'\i as de haskell que brindan facilidades para el tratamiento musical.

\end{document}
