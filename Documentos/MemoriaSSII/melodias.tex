\chapter{Generador de melod\'ias}
\section{Introducci'on}
El m'odulo generador de melod'ias se encarga de para componer una melod'ia a partir de una progresion de acordes y un patr'on r'itmico. Se utilizan las reglas de la armon'ia para que la melod'ia resultado este relacionada con la progresi'on y `quede bien` al sonar a la vez que la traducci'on a m'usica de 'esta.
\newline
Todas las melod'ias compuestas por Genaro son monof'onicas, esto es, son sucesiones de notas en el tiempo tales que no suena m'as de una nota a la vez. Las notas no se superponen en el tiempo como en el caso de los acompa~namientos. Esta simplificaci'on facilita mucho el proceso de composici'on.

\section{Abstracciones empleadas}
La generaci'on de melod'ias de Genaro se basa en dos abstracciones, una \emph{curva mel'odica} y una \emph{lista de acentos}:
        \begin{enumerate}
        \item La curva mel'odica: es una lista de enteros que se entienden como saltos dentro de una escala que no se ha fijado todav'ia.
                \begin{itemize}
                \item Un salto de 0 indica quedarse en la misma nota, un salto de n con n distinto de 0 indica moverse n notas dentro de una escala, hacia arriba en el caso de que n sea positivo y hacia abajo si es negativo.
                \item Una curva mel'odica se refiere por tanto a las alturas de las notas solamente, no indica nada referido a duraciones ni al ritmo.
                \item Una curva melod'ica representa las relaciones de altura relativa entre las notas que conforman la melod'ia, y en ning'un momento se refiere a alturas absolutas. En un paso determinado del proceso de generaci'on de melod'ias se realiza la correspondencia entre una curva mel'odica y una lista de alturas, especificando una escala y una altura de partida que fija una referencia absoluta desde la que comenzar el proceso.
                \item Al pasar a la lista de alturas es posible que la altura de partida no pertenezca a la escala. En ese caso se entender'a que en el primer punto de la curva saltamos a una altura de la escala.
                \end{itemize}
        \item La lista de acentos: es una lista de parejas (float, fraccion entera) que expresa un perfil r'itmico de la melod'ia.
                \begin{itemize}
                \item Cada elemento de la lista de acentos representa la duraci'on de una posible nota (segunda componente) y la fuerza con la que 'esta se atacar'ia por parte del instrumentista que interpretara la melod'ia (primera componente):
                        \begin{enumerate}
                        \item[(a)] Las duraciones se miden como figuras musicales.
                        \item[(b)] La intensidad con que se ataca cada nota se mide de 0 a 100. Los casos de intensidades que valen 0 corresponden a los silencios, componentes fundamentales de la m'usica.
                        \end{enumerate}
                \item Las listas de acentos se construyen de forma determinista procesando patrones r'itmicos pero su aplicaci'on no es absoluto determinista, como se ver'a m'as adelante. Se pueden entender como la correspondencia de una dimensi'on de los patrones r'itmicos, que tienen dos dimensiones.
                \end{itemize}
        \end{enumerate}

Otra abstracci'on menos original pero tambi'en importante es el concepto de \emph{registro}. Un registro es una restricci'on sobre las alturas de las notas, que deben pertenecer a un intervalo determinado por 'este. Asignamos un registro aguda a la melod'ia para que resalte sobre el resto de pistas. El bajo a su vez tendr'a asociado un registro grave, como es l'ogico.

\section{Algoritmo de generaci'on de melod'ias}
Utiliza las abstracciones anteriores. Para saber obtener una melod'ia para una progresi'on primero debemos aprender a obtener una melod'ia para un solo acorde, y luego a enlazar las melod'ias de cada acorde.
\newline
El algoritmo que hace una melod'ia para un solo acorde recibe de entrada una curva mel'odica, una lista de acentos, el acorde sobre el que tiene que construir la melod'ia, la altura desde la que empezar la melod'ia y una serie de par'ametros enteros cuya utilidad explicaremos despu'es. Partiendo esa informacion aplicamos los siguientes pasos:
        \begin{enumerate}
        \item Ajustar curva mel'odica con lista de acentos: se modifica la lista de acentos para que tenga la misma longitud que la curva aleatoria. Para ello se dividen en dos o se unen parejas de acentos, seg'un la lista de acentos sea m'as corta o m'as larga que la curva mel'odica. Los acento que se dividen o fusionan se eligen aleatoriamente.
        \item Elegir los acentos: se elige de forma aleatoria una sublista de tama~no aleatorio de la lista de acentos, dando mayor probabilidad de ser elegidos a los acentos m'as fuertes. Tambi'en se guardan en otra lista los acentos rechazados.
        \item Elegir los puntos de la curva mel'odica: se hace la correspondencia entre la curva mel'odica y una lista de alturas de nota empleando la escala correspondiente al acorde, la altura de partida, y el registro correspondiente a la melod'ia. Estas alturas concretas ya pueden ser valoradas como m'as o menos estables en el contexto arm'onico que establece el acorde. Se elige una sublista aleatoria de esta lista de alturas, de la misma longitud que la sublista de acentos anterior, dando mayor probabilidad de ser elegidas a las alturas m'as estables.
        \item Construir la primera versi'on de la melod'ia: a partir de las tres sublistas anteriores se construyen dos listas, una a partir de los acentos y alturas elegidos, la lista de notas que suenan; y otra a partir de la lista de acentos rechazados, la lista de silencios o notas que no suenan. Estas dos listas se mezclan siguiendo el orden de los acentos en la lista de acentos original, y se obtiene una primera versi'on de la melod'ia.
        \item Mutar la melod'ia: el resto del proceso consiste en aplicar mutaciones sucesivas a la primera versi'on de la melod'ia. Estas mutaciones se aplican siempre en el mismo orden y seg'un indican los 4 enteros de entrada al algortimo.
                \begin{enumerate}
                \item[i)] Alargar notas: considera todas las notas de la melod'ia que no sean silencios y a las que siga inmediatamente un silencio. Elige aleatoriamente un n'umero aleatorio de estas notas y alarga su duraci'on con la del silencio que les segu'ia, elimin'andolo. Se aplica tantas veces como indique el primer entero.
                \item[ii)] Dividir notas: considera todas las notas de la melod'ia que no sean silencios y a las que siga una nota que no sea un silencio, aunque no sea inmediatamente (aunque haya silencios entre estas dos notas no silencio). Elije aleatoriamente una sola de estas notas y elige aleatoriamente una altura entre la de la nota elegida y la de la primera nota no silencio que le sigue en la melodia. Despu'es elimina los silencios entre estas dos notas e en su lugar inserta una nueva nota y silencios de forma que la nueva nota tiene la altura elegida anteriormente y est'a a la misma distancia en el tiempo de las dos notas. En resumen, calcula la distancia en el tiempo entre dos notas e introduce una nota entre otras ellas, en la mitad de la distancia. Se aplica tantas veces como indique el segundo entero.
                \item[iii)] Dividir notas fino: es como dividir notas solo que en vez de dividir por dos divide por una potencia de dos. Se aplica tantas veces como indique el tercer entero, y en cada aplicaci'on divide por una potencia de dos de exponente elegido aleatoriamente entre 0 y el cuarto entero.
                \end{enumerate}
        \end{enumerate}

Una vez ya sabemos construir una melod'ia para un acorde construir una melod'ia para una progersi'on es sencillo. El algoritmo recibe como entrada una progresi'on, un patr'on ritmico y una serie de par'ametros enteros entre los que se incluyen los anteriores. Partiendo esa informacion aplicamos los siguientes pasos:
        \begin{enumerate}
        \item Generar la curva mel'odica: Genaro tiene un algoritmo de generaci'on aleatoria de curvas mel'odicas. Se le especifica el n'umero de puntos deseado, el salto m'aximo (valor absoluto m'aximo de un punto de la curva) y un peso para el cambio de direcci'on de la curva (la curva cambia de direcci'on cuando pasa de hacer saltos positivos a hacer saltos negativos o vicecersa). El algoritmo genera una curva que cumple esas restricciones, y para ello parte de un primer punto que vale 0, y de una direcci'on de movimiento (arriba o abajo) elegida aleatoriamente. Para el resto de puntos elige aleatoriamente si cambia de direcci'on o no teniendo en cuenta el peso suministrado y el tiempo que lleva `caminando` en una direcci'on, cuanto m'as rato lleve mayor es la probabilidad de que cambie de direcci'on con un salto mayor. Si no cambia de direcci'on son m'as probables los saltos peque~nos. Seg'un estas probabilidades e intervalos de valores v'alidos va generando aleatoriamente los puntos de la curva.
        \item Distribuir la curva entre los acordes: La curva mel'odica se genera para toda la progresi'on por lo que debe distribuirse entre los acordes de la progresi'on. A cada acorde le corresponde una cantidad de puntos de la curva seg'un su duraci'on, dando m'as puntos a los acordes m'as largos. A cada acorde le corresponde un porcentaje del n'umero de puntos de la curva igual a su duraci'on entre la duraci'on total de la progresi'on. Como esos porcentajes multiplicados por el n'umero de puntos de la curva no suelen dar lugar a enteros se distribuye la parte entera y lo que sobra se reparte resolviendo aleatoriamente los conflictos. Una vez decidido cuantos puntos corresponden a cada acorde estos se reparten en orden, es decir, que si a cada acorde i le corresponden $n_{i}$ puntos, entonces al acorde 1 le corresponden los puntos del 1 al $n_{i}$, al acorde 2 del $n_{i}$ + 1 al 2 * ($n_{i}$), etc...
	\item Construir la lista de acentos a partir del patr'on r'itmico: Esto se hace de forma determin'ista, primero se calcula la lista formada por la media de los acentos de cada columna. Luego se calcula la media de estos acentos medios. Por 'ultimo se procesa la lista de acentos medios de forma que los acentos mayores o iguales que la media se dejan igual, y los menores que la media se dejan a cero. Finalmente se empareja esta 'ultima lista con las duraciones correspondientes a cada columna. Esta lista de acentos ser'a la misma para todos las melod'ias de todos los acordes de la progresi'on, de la misma forma que un patr'on r'itmico es com'un a todos los acordes de una progresi'on.
	\item Construir las melod'ias para cada acorde: Primero se genera aleatoriamente una altura de partida para el primer acorde, dando una probabilidad mayor de ser elegidas a las notas m´as estables de 'este. Con esta altura de partida, la curva mel'odica que le corresponde, la lista de acentos, el primer acorde y los par'ametros correspondientes se genera la melod'ia para el primer acorde. Las melod'ias de los acordes siguientes se construyen igual, pero tomando como nota de partida la 'ultima nota de la melod'ia del acorde anterior.
	\item Componer las melod'ias: Las melod'ias resultado se componen secuencialmente.
        \end{enumerate}


\section{Implementaci'on}
El m'odulo generador de melod'ias esta desarrollado 'integramente en Haskell, y se apoya en la librer'ia Haskore \cite{hudak} para componer melod'ias sobre un acompa~namiento.
\newline
La comunicaci'on de este m'odulo con el generador de progresiones se realiza a trav'es de los ficheros de texto que escribe el generador de progresiones. Estos ficheros son leidos por Haskell y procesados por un parser generado a partir de la librer'ia \cite{fokker}, para obtener elementos de tipo Progresion, un tipo definido en Haskell como sigue:
        \begin{verbatim}
data Grado = I|BII|II|BIII|III|IV|BV|V|AUV|VI|BVII|VII|VIII
             |BBII|BBIII|AUII|BIV|AUIII|AUIV|BBVI|BVI|AUVI|BVIII|AUVIII
             |BIX|IX|BX|X|BXI|XI|AUXI|BXII|XII|AUXII|BXIII|XIII|AUXIII
             |V7 Grado
             |IIM7 Grado
     deriving(Show,Ord)

instance Eq Grado where
  g1 == g2 = (gradoAInt g1::Int) == (gradoAInt g2::Int)

data Matricula = Mayor|Menor|Au|Dis|Sexta|Men6|Men7B5|Maj7|Sept|Men7
                |MenMaj7|Au7|Dis7
     deriving(Enum,Read,Show,Eq,Ord,Bounded)

type Cifrado = (Grado, Matricula)

type Progresion = [(Cifrado, Dur)]
        \end{verbatim}
Tambi'en se recurre a la librer'ia \cite{fokker} para leer los ficheros de patr'on r'itmico escritos por el editor de C ++.
\newline

El resto de manipulaciones se realizan utilizando los tipos de Haskore, las abstracciones anteriormente introducidas, y otros tipos que representan informaci'on sobre escalas y otra serie de informaci'on sobre teor'ia musical. Todos estos conceptos est'an representados en Haskell como sigue:
        \begin{verbatim}
type SaltoMelodico = Int
type CurvaMelodica = [SaltoMelodico]
type CurvaMelodicaProgresion = [CurvaMelodica]
type ListaAcentos = [(Acento, Dur)]
type Acento = Float 
{-
Identificadores de cada tipo de escala
-}
data Escala = Jonica|Dorica|Frigia|Lidia|Mixolidia|Eolia|Locria
             |MixolidiaB13|MixolidiaB9B13AU9|Disminuida
     deriving(Enum,Read,Show,Eq,Ord,Bounded)
{-
InfoEscala = (tensiones, lista de notas que la forman, notas a evitar)
-}
type InfoEscala = ([Grado], [Grado], [Grado])
{-
Indica cuales son las octavas apropiadas para cada instrumento
-}
type Registro = [Octave]
        \end{verbatim}

Las melod'ias como producto 'ultimo se representan con la constructora Music de Haskore. Pero la entrada a las funciones de mutaci'on es una lista de elementos de tipo Music, que representan cada uno notas individuales. Esta representaci'on intermed'ia es m'as f'acil de manipular, y se transforma facilmente a la representaci'on final, aplicando la operaci'on de composici'on secuencial de Haskore.

\section {Otros usos de 'este m'odulo}
La interfaz gr'afica desarrollada en C++ ofrece al usuario un acceso sencillo a funcionalidades m'as sofisticadas. Las funcionalidades que ofrece de este m'odulo conbinado con el interfaz gr'afico son:
	\begin{itemize}
	\item Manipulaci'on de curvas mel'odicas: Se pueden manipular las curvas mel'odicas de diversas maneras:
		\begin{enumerate}
		\item[(a)] Salvado y carga de curvas mel'odicas. Esto permite aplicar la misma curva mel'odica en diversas pistas y bloques de la composici'on. Como la aplicaci'on de las curvas mel'odicas no es determinista, esto se puede entender como hacer variaciones sobre un tema com'un.
		\item[(b)] Existe un editor manual de curvas mel'odicas, con el que el usuario puede dise~nar curvas mel'odicas, visualiz'andolas como una linea de puntos en dos dimensiones, lo que facilita la comprensi'on intuitiva de su significado. Para que el usuario comprenda lo que hace al disen~ar una curva, y lo que significan las curvas que hace Genaro.
		\item[(c)] Existe una funci'on que muta aleatoriamente curvas mel'odicas, cambiando tantos puntos como se le indique, dentro de un intervalo de variaci'on determinado.
	\end{enumerate}
	\item Copia de melodias: Se puede copiar exactamente una melodia de una pista y subbloque determinado en otra pista y subbloque concretos. Esto es 'util si se quiere repetir una melod'ia para rearmonizarla.
	\item M'odulo de bajo: se basa fuertemente en este m'odulo, tanto en funciones a las que llama como en las ideas de sus algoritmos.
	\end{itemize}

\section {Capacidad de ampliaci'on}
La ausencia de objetos e interfaces y las abstracciones definidas favorecen la ampliabilidad de este m'odulo, proporcionando puntos de acceso para la generaci'on de melod'ias. Las vias m'as evidentes por las que se podr'ia ampliar este m'odulo son:
	\begin{enumerate}
	\item Dise~nar nuevas mutaciones para la melod'ia. O permitir que se apliquen en distinto orden, o permitir mutar las melod'ias despu'es de ser creadas.
	\item Implementar un editor manual de melod' ias.
	\item Si se ampl'ia el m'odulo generador de acordes para que trabaje con nuevos tipos de acordes se deber'ia ampliar tambi'en este m'odulo para que trate bien esos nuevos acordes. Por la estructura de 'este esta ampliaci'on ser'ia muy sencilla.
	\item Extrapolaci'on de las curvas mel'odicas a partir de una melod'ia renderizada a midi.
	\item Algo totalmente diferente: Cualquier programa en cualquier lenguaje que lea archivos de texto en el formato especificado por el predicado es\_progresion/1, y ficheros de patr'on r'itmico puede sustituir a este m'odulo.
	\end{enumerate}

Quiz'as la aplicacion de la curva melodica es demasiado poco determinista como para hablar de variaciones al aplicarla varias veces o mutarla. Un concepto menos abstracto intermedio podr'ia ser m'as apropiado. Ese concepto podr'ia ser simplemente las listas de notas representadas por Haskore que son la entrada de los procedimientos de mutaci'on. Si se implementara el acceso por interfaz a la mutaci'on de melodias podr'iamos hablar con m'as justificaci'on de variaciones sobre una melod'ia.\newline
Y otra ampliaci'on que ser'ia muy interesante es introducir simetr'ias y estructuras como la frase cuadrada (motivo-contramotivo-motivo-resoluci'on) en la generaci'ion de melod'ias. Genaro producir'ia melod'ias muy peque~nas que las que luego dar'ia m'as coherencia y l'ogica ajustandolas a estas estructuras. Para ello se podr'ian utilizar las mismas ideas de variaciones sobre una melod'ia comentadas en el p'arrafo anterior.
