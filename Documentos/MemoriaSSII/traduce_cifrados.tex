\documentclass[a4paper]{report}
\usepackage[spanish]{babel}
\usepackage{musixtex}
\title{Traduccion de una lista de cifrados}
\author{Roberto Torres de Alba}
\begin{document}
\maketitle
\section{Introduccion}
La finalidad del m\'odulo de traducci\'on de cifrados es que proporcione todas las notas
exactas que poseer\'a cada cifrado en funci\'on, por supuesto, de ciertas opciones de 
entrada y, en caso de ambig\"uedad, usara la aleatoriedad para elegir entre las posibles
opciones.\\
\indent Cuando nos referimos a las notas exactas de un cifrado musical nos referimos a las
alturas de las notas, tanto el \emph{PitchClass} como la octava, es decir, aquello que determina la 
frecuencia de los sonidos. No nos referimos a como esas notas se van a ejecutar en el tiempo 
porque eso es parte de la siguiente etapa en la que vamos a encajar estas notas con lo que hemos
llamado patron r\'\i tmico. Por ello no es necesario tener en cuenta la duraci\'on individual de 
cada nota, que ser\'\i a lo \'unico que nos faltar\'\i a para tener una verdadera nota musical, 
sino que basta tener la duraci\'on del acorde. Es por ello que la salida de este modulo es el tipo: \\
\begin{verbatim}
type AcordeOrdenado = ([Pitch],Dur)
\end{verbatim}
es decir, la duracion del acorde junto con las notas que tiene.\\
\indent Pongamos un ejemplo. El cifrado de C Maj7 (do mayor con s\'eptima mayor), posee las notas
(C,E,G,B) pero esa informacion esta incompleta para una traduccion verdadera. Supongamos
que buscamos que ese acorde tenga 7 notas, est\'e en estado fundamental (el C como nota m\'as
grave) y en primera disposici\'on (el C como nota m\'as aguda). Es obvio que tenemos que 
duplicar algunas alturas para que se cumpla lo anterior. El programa actuara devolviendo lo
siguiente: (C,E,G,B,C,G,C). Ahora s\'\i~que se cumple lo anterior pero la traducci\'on sigue 
siendo incompleta seg\'un lo dicho anteriormente ya que falta la octava. Digamos que ahora queremos que la nota
m\'as grave est\'e en la cuarta octava y las dem\'as notas sean la menor siguiente m\'as aguda. Entonces el acorde quedar\'\i a como sigue:
((C,4),(E,4),(G,4),(B,4),(C,5),(G,5),(C,6)) que s\'\i ~es lo que buscamos.\\
\indent La etapa de traducci\'on se compone de dos fases:\\
1. Traducci\'on a forma fundamental: simplemente indica los emph{PitchClass} que posee
el acorde\\
2. Reparto de las notas en voces: duplica y permuta los emph{PitchClass} de la etapa 1
para que se ajusten a unos determinados par\'ametros. Vamos adelantando que los par\'ametros
m\'as importantes son los dos sistemas de traducci\'on: el continuo y el paralelo.\\
% falta decir que esta hecho en haskell

\section{Traducci\'on a forma fundamental}
\subsection{Vectores de matr\'\i culas}
Un cifrado se compone de dos partes: el grado sobre el que se asienta y la especie
que es. Nosotros hemos llamado a la especie emph{matr\'\i cula}, de esa forma, por ejemplo, 
el cifrado de emph{C Maj7} posee de matr\'\i cula emph{Maj7} y de vector de emph{PitchClass} [C,E,G,B]; el cifrado
de emph{C m} (do menor) posee matricula emph{m} y emph{PitchClass} [C,bE,G].\\
\indent Independientemente del grado del cifrado todos los acordes que son mayores posee
una tercera mayor y una quinta justa (un m\'usico entender\'\i a esto perfectamente aunque
no es necesario para lo que nos trata a continuaci\'on). Una tercera mayor est\'a compuesta
de 4 semitonos (\'o 2 tonos) y una quinta justa de 7 semitonos (\'o 3 tonos y medio). De esa
forma es f\'acil ver que para obtener un acorde mayor sobre cualquier nota simplemente
tenemos que sumar 4 semitonos a dicha nota (obteniendo la tercera del acorde) y de nuevo
7 (para obtener la quinta). M\'as f\'acil es ver todav\'\i a que eso se puede expresar en forma
de tupla como (0,4,7). Veamos un ejemplo para aclararlo:\\
Buscamos el acorde de D mayor. El emph{PitchClass} D en emph{Haskore} se representa por el n\'umero 2, 
cosa que nos facilita la suma. Si sumamos 2 + (0,4,7) nos da (2,6,9) que pas\'andolo otra vez
a notas nos da (D,Fs,A) (Fs es Fa sostenido). Aqui hay que decir que tanto emph{Haskore} como MIDI usa un sistema
atemperado, es decir, que las notas Fs y Gf (Sol bemol) representan el mismo sonido.
Una ver entendido este concepto pasamos a mostrar todos los vectores asociados con
las matr\'\i culas que maneja Genaro:\\
Mayor $\to$ [0,4,7]\\
Menor $\to$ [0,3,7]\\
Au $\to$ [0,4,8]\\
Dis $\to$ [0,3,6]\\
Sexta $\to$ [0,4,7,9]\\
Men6 $\to$ [0,3,7,9]\\
Men7B5 $\to$ [0,3,6,10]\\
Maj7 $\to$ [0,4,7,11]\\
Sept $\to$ [0,4,7,10]\\
Men7 $\to$ [0,3,7,10]\\
MenMaj7 $\to$ [0,3,7,11]\\
Au7 $\to$ [0,4,8,10]\\
Dis7 $\to$ [0,3,6,9]\\
Las funciones de Haskell que encapsulan esto se encuentran en el modulo emph{Progresiones.hs}
y se llaman emph{matriculaAVector} y emph{sumaVector}.\\
\subsection{Traducci\'on a forma fundamental}
Una vez que tenemos a nuestra disposici\'on los vectores de las matr\'\i culas ya podemos
traducir cualquier cifrado a su forma fundamental:\\
\begin{verbatim}
traduceCifrado :: Cifrado -> [PitchClass]
traduceCifrado (grado, matricula) = map  absPitchAPitchClass (sumaVector (matriculaAVector matricula) (gradoAInt grado))
\end{verbatim}

\section{Reparto de las voces}
Existen dos formas principales de ordenas las notas de la etapa anterior: sistema paralelo
y sistema continuo. Ambas formas han sido inspiradas del libro de Enrik Herrera.\\
\subsection{Sistema Paralelo}
El sistema paralelo es el mas simple. Basicamente se encarga de colocar todos los acordes
en una inversion y una disposicion dada. Los parametros que necesitan que sean pasados 
por el usuario son los siguientes: \\
octava inicial: en la octava a partir de la cual empieza a color las notas\\
inversion: un entero que indica la inversion que es\\
disposicion: otro entero que indica la disposicion del acorde\\
numero de notas: numero de notas final del acorde\\
La inversion de un acorde es simplemente la indicacion de que nota es la mas grave en un acorde.
Una inversion de 1 indica que el acorde esta en primera inversion (la tercera del acorde 
como nota mas grave) y asi. La inversion 0 se usa para decir que estamos en estado fundamental
 (la fundamental del acorde como nota mas baja).\\
Algo parecido para con la disposicion que unicamente depende de la nota mas aguda.
Una disposicion de 1 indica que la fundamental del acorde es la nota mas alta y asi.\\
La inversion como la disposicion depende unicamente de la nota
mas grave y de la mas aguda y eso implica dos cosas. Primera es que el numero minimo de
notas que puede tener un acorde es dos (las dos notas extremas). Se podria haber hecho de
otra forma pero esta es la que vimos que coherente con los parametros. Segundo es que
no hay informacion sobre la colocacion del resto de las notas. Podriamos haber seguido
una opcion parecida a la del sistema continuo (como ya veremos) pero pensamos que iba
mas encaminado con la filosofia de este sistema (segun Enrik Herrera, por supuesto)
que las notas intermedias se colocaran en el mismo orden que establece sus estado fundamental.\\
Una nueva abstraccion tenemos que introducir en este momento: el AcordeSimple.\\
\begin{verbatim}
type AcordeSimple = [Int] 
\end{verbatim}
Un acorde simple es solamente una lista de numeros. Cada numero es un indice sobre la lista de
notas del acorde que se pasa de la etapa anterior, es decir, sobre un acorde en estado 
fundamental. Para formar un acorde simple segun los parametros del sistema paralelo usamos\\
\begin{verbatim}
formarAcordeSimple :: NumNotasFund -> NumNotasTotal -> Inversion -> Disposicion -> AcordeSimple
formarAcordeSimple numNotasFund numNotasTotal inv disp = reverse (disp : anadeAcordeSimple numNotasFund (numNotasTotal - 2) [inv + 1])
\end{verbatim}
�Para que nos sirve esto? Simplemente independiza las notas verdaderas del acorde con su
ordenacion. Esto se entiendo muy facilmente con un ejemplo:\\
Supongamos el acorde de C Mayor de nuevo cuyo estado fundamental es [C,E,G] que es proporcionado
por la etapa 1. Queremos este acorde con 10 notas, en primera inversion, y en tercera
disposicion. Llamando a la funcion formarAcordeSimple tenemos:\\
\begin{verbatim}
TraduceCifrados> formarAcordeSimple 3 10 1 3 
[2,3,1,2,3,1,2,3,1,3] 
\end{verbatim}
�Que significa esto? Significa que el primer elemento es el segundo elemento de [C,E,G],
el segundo es el tercer elemento de [C,E,G] y asi. La importancia de esto es que lo
que hemos llamado acorde simple es independiente de las notas concretas.\\
Ahora simplemente hay que fusionarlas dos cosas pero eso es muy facil ya que son indices
(recordemos que con el 1 nosotros nos referimos al primer elemento, a diferencia de en
Haskell que se refiere al primer elemento con el 0)
\begin{verbatim}
encajaAcordeSimple :: [PitchClass] -> AcordeSimple -> [PitchClass]
encajaAcordeSimple lpc [] = []
encajaAcordeSimple lpc (num : resto) = (lpc !! (num - 1)) : encajaAcordeSimple lpc resto
\end{verbatim}
Ahora solo nos queda la parte de anadir la octava.

\subsection{Sistema Continuo}
La idea principal de este sistema es que las voces de un acorde se muevan lo menos
posible respecto a las voces del acorde anterior, de esa forma el efecto transitorio entre
acordes sera mas suave.\\
Mas detalladamente se busca lo siguiente: si un acorde posee notas en comun respecto del anterior entonces esas notas tienen que 
ser la misma; el resto de las voces se mueve a la nota mas cercana.\\
Enrik Herrera propone en la pagina 42 un metodo que nos hace saber cuantas notas en comun 
poseen esos dos acordes. En primer lugar ese metodo es solo para triadas aunque se podria
ampliar sin mucha dificultad a cuatriadas. Por otro lado solo tiene en cuenta los acordes
diatonicos. Eso es un problema si queremos usar algo parecido sobre acordes no diatonicos
como puede ser los dominantes secundarios. Por tanto este metodo no es suficientemente 
general para nuestros propositos y tenemos que buscar otro.\\
El algoritmo se divide en cuatro partes. En la primera traducimos todos los cifrados
a su estado funtamental sin duplicar ni permutar los PitchClass, es decir, lo que optenemos
directamente de usar el Vector de la matricula. En la segunda permutamos cada acorde
hasta que la diferencia de las notas sea la mas peque�a con respecto al acorde anterior. 
En la tercera duplicamos las notas para que posean el numero de notas deseadas. Y en
la cuarta insertamos la octava para que la altura de las notas este fijada definitivamente.
Ahora vamos a verlos en detalle.

\subsubsection(Traduccion a estado fundamental)
La traduccion a estado fundamental no merece demasiada atencion ya que se realiza de la misma
forma a la del sistema paralelo, usando la funcion emph{traduceCifrado}.

\subsubsection(Obtencion del acorde mas cercano al anterior)
Esta es, segun mi criterio, la parte mas importante del algoritmo del sistema continuo.
La base del algoritmo es bastante facil. En primer lugar necesitamos dos operaciones
sobre listas de PitchClass: una que nos diga las notas en comun en la misma posicion
de la lista y otra que nos diga la distancia entre dos listas de PitchClass. Esto nos da
lo siguiente: \\

\begin{verbatim}
coincidencias :: [PitchClass] -> [PitchClass] -> Int
coincidencias lp1 lp2 = foldr1 (+) (map fromEnum [(lp1 !! i) == (lp2 !! i) | i <- [0..(longMen - 1 )]])
	where longMen = min (length lp1) (length lp2)

distancia :: [PitchClass] -> [PitchClass] -> Int
distancia lp1 lp2 = foldr1 (+) [abs ( ((map pitchClass lp1) !! i) - ((map pitchClass lp2) !! i) ) | i <- [0..(longMen - 1 )] ]
	where longMen = min (length lp1) (length lp2)
\end{verbatim}

Una vez que tenemos estas dos funciones ya podemos sacar el acorde mas cercano a otro de 
la siguiente forma. Primero hacemos todas las permutaciones de la lista de PitchClass que
representa el acorde en posicion fundamental. Eso no es tan preocupante computacionalmente 
hablando ya que nosotros vamos a tener como maximo cuatriadas que forman listas de cuatro
elementos (4\! combinaciones). Segundo vamos a sacar los acordes cuya funcion coincidencias
nos de el mayor valor y eliminar los otros. Tercero vamos a sacar los acordes (de los que quedan)
cuya funcion distancia al anterior sea la menor. Cuarto de los restantes vamos a elegir uno al azar.
Muy posiblemente ya en la segunda etapa solo nos quede una lista de un elemento aunque
hay que asegurarse hasta el final.\\
El primer acorde la progresion es un caso especial ya no tiene ningun acorde anteriormente con
el que poder compararse. Lo que hacemos es simplemente elegir una inversion aleatoria
para el.\\
El algoritmo completo se recoge en la funcion organiza que es: \\
\begin{verbatim}
organizar :: RandomGen a => a -> [[PitchClass]] -> [[PitchClass]]
organizar gen (x:xs) = elegido : organizarRec sigGen elegido xs
    where lista = [inversion i x | i <- [0..(length x - 1 )]];
          (elegido, sigGen) = elementoAleatorio gen lista

organizarRec :: RandomGen a => a -> [PitchClass] -> [[PitchClass]] -> [[PitchClass]]
organizarRec _ _ [] = []
organizarRec gen referencia (x:xs) = nuevaRef : organizarRec sigGen nuevaRef xs
    where (nuevaRef, sigGen) = masCoincidente gen referencia x

\end{verbatim} 
donde inversion es la i-esima inversion de la lista x.

\subsubsection(Duplicacion de notas)
La idea es la de usar el llamado AcordeSimple para acceder a las notas ya ordenadas de los acordes. Ahora todos
los acordes simples se van a forman empezando desde el numero 1. Eso no implica que el acorde vaya a estar en forma
fundamental ya que el indice 1 sobre la lista de PitchClass no corresponde a la fundamental del acorde porque la
etapa anterior se ha encargado de permutar la lista adecuadamente. El codigo del este algoritmo se encuentra
integrado en al misma funcion que la del siguiente por lo que se vera despues como es.

\subsubsection(Insercion de octava)
Ahora solamente falta insertar la octava en las listas de PitchClass para que tengan la altura correspondiente.
Suponemos que el acorde anterior ya tiene todas las octavas a~nadidas y buscamos introducirlas al actual.
No podemos hacer que empiecen a a�adirse desde la mas grave como se hizo en el sistema paralelo porque eso podria
darnos alturas diferentes. Hay que hacer lo siguiente: buscar la primera nota coincidente entre los dos acordes, 
dividir el acorde actual por dos por dicha nota y tomar la octava de esa nota. Una vez dividido la parte mas grave
del acorde se le asigna octavas de forma descendente y la mas aguda de hace de forma creciente. Como las notas comunes
estan en la misma posicion dentro de las listas de PitchClass no asegura que dichas notas comunes reciben la misma octava


\end{document}









