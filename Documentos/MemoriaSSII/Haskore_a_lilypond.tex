\documentclass[a4paper]{report}

\usepackage[spanish]{babel}
\usepackage{musixtex}
\title{Haskore a Lilypond}
\author{Roberto Torres de Alba}
\setlength{\hoffset}{-2cm}
\setlength{\voffset}{-2cm}
\setlength{\textwidth}{450pt}

\begin{document}
\maketitle
\chapter{Haskore a Lilypond}

\section{Introduccion}

%hablar del cygwin
%hablar de sostenidos y bemoles
Este modulo pretende ser un traductor que pasa del arbol que representa 
el tipo Music a un String que sea entendido por lilypond.\\
Nosotros hemos usado la version 2.4.3 de lilypond para compilar.\\
En primer lugar ibamos a usar el programa midi2ly que viene junto con 
lilypond que pasa un archivo midi a formato lilypond. Las pruebas que hicimos
con ellos nos desilusiono bastante ya que el resultado obtenido estaba lejos
de ser correcto a pesar de que nuestros ejemplos eran muy simples 
y hechos con Haskore, es decir, que no estaba la mano de ningun humano
que podia varia el tiempo en milesimas.\\
Es por ello que intentamos hacer la traduccion nosotros mismos a algun
formato que se pudiera compilar a pdf. Vimos que, a diferencia de otros
como musicTex o musixTex, el formato de lilypond era muy parecido a la
estructura del tipo Music de Haskore. Esto nos motivo a hacer la conversion 
de esta forma.

\section{Traduccion}
En esta seccion no vamos a entrar en como es el proceso de conversion
completo ni en todas las caracteristicas de lilypond porque, mas que nada,
ni yo mismo las conozco.\\
Ya que el tipo Music no posee informacion sobre la obra completa hemos
tenido que hacer un tipo que lo englobe. Ese tipo se llama CancionLy
y es muy simple de entender:
\begin{verbatim}
type Armadura = (PitchClass, Modo)
data Modo = Mayor | Menor
     deriving (Eq, Ord, Show, Read, Enum)
type Ritmo = (Int  -- Numero de notas del compas
             ,Int  -- Resolucion del compas
             )
type Instrumento = String
data Clave = Sol   -- Clave de 'Sol'
             | Fa  -- Clave de 'Fa'
             | Bateria -- Clave de 'Bateria'
type Score = (Music, Armadura, Ritmo, Instrumento, Clave)
type Titulo = String
type Compositor = String

type CancionLy = (Titulo, Compositor, [Score])
\end{verbatim}
Muchas de las cosas anteriores se entienden por lo que no vamos a entrar
a detallarlas. La parte mas importante es la traduccion del tipo Music
que es parte de un Score (pentagrama en ingles).

\subsection(Traduccion del tipo Music)
Como hemos dicho, lilypond y Music tienen varias cosas en comun que
hace atractiva su conversion. En concreto son dos: ambos poseen
un operador para secuenciar musica y para paralelizarla.\\
El operador de musica paralela (que se interpreta a la vez) es el :=:
en Haskore. En lilypond es todo aquello que va entre $<<$ y $>>$. El 
operador secuencia en Haskore es el :+: y en lilypond es todo aquello
que no esta entre $<<$ y $>>$. Lilypond tambien es capad de agrupar
musica poniendo llaves \{ \}.
\begin{verbatim}
deMusicALy :: Music -> String
deMusicALy (Note p dur _) = imprimeNota p dur
deMusicALy (Rest dur)     = imprimeSilencio dur
--   Cambia el operador secuencia de Haskore por el secuencia de Lilypond
deMusicALy (m1 :+: m2)    = " { " ++ deMusicALy m1 ++ " " ++ deMusicALy m2 ++ " } "
--   Cambia el operador paralelo de Haskore por el paralelo de Lilypond
deMusicALy (m1 :=: m2)    = " << " ++ deMusicALy m1 ++ " " ++ deMusicALy m2 ++ " >> " 
deMusicALy (Tempo d m)    = "\\times " ++ imprimeRatio d ++ " { " ++ deMusicALy m ++ " } "
--   Elimina la constructora 'Trans'
deMusicALy (Trans t m)    = deMusicALy (suma t m)
--   El resto de las constructoras son ignoradas
deMusicALy (Instr _ m)    = deMusicALy m
deMusicALy (Player _ m)   = deMusicALy m
deMusicALy (Phrase _ m)   = deMusicALy m
\end{verbatim}
Esta es la funcion central del modulo HaskoreALilypond.hs. Algunas constructoras
de Music tiene que ser ignoradas porque o no corresponden con nada de
lilypond o la traduccion seria tan complicada que llevaria mucho tiempo y no 
mereceria la pena.\\
Por ultimo hay que se~nalar un apunte sobre la traduccion de duraciones de
notas o silencios. Lilypond solo entiende duraciones que son potencias de
dos, es decir, 1 es la redonda, 2 la blanca, etc. Pero las duraciones en
Haskore son fracciones de enteros. Por ello se ha hecho de la siguiente forma:\\
1. Si el numerador es mayor que uno entonces se repite tantas veces como diga
dicho numerador y todas las notas se ligan.\\
2. Si el denominador no es potencia de 2 se cambia a la potencia de dos mas
cercana. Este ultimo paso elimina informacion del tipo Music pero no se que
podriamos hacer si llega algo por el estilo de 1/27. En caso de dejarselo 
a lilypond este siempre pone una redonda en caso de no entenderlo, cosa que
es todavia peor.\\
Esto ha motivado la siguiente funcion:
\begin{verbatim}
imprimeNota :: Pitch -> Dur -> String
imprimeNota p dur 
   | numerador == 3 && denominador > 1 = imprimePitch p ++ show (quot denominador 2) ++ "." -- Ponemos puntillo
   | otherwise                         = eliminaUltimos 2 (concat [imprimePitch p ++ show (redondeaAPotenciaDos(denominator dur)) ++ "~ " | i <- [1..numerador] ] ) -- elimina los dos ultimos para acabar con el ultimo "~"
   where numerador = numerator dur;
         denominador = denominator dur;
\end{verbatim}

\section{Problemas}
El principal problema no ha sido la conversion al formato lilypond. Sabemos
que muchas veces, sobre todo en melodia y bajo, el denominador de las notas 
no va a ser potencia de dos. nos hubieramos conformado con que tuvieramos
una ligera aproximacion a las notas que genera Genaro.\\
El principal problema es el propio lilypond. La version 2.4.2 estaba mal 
implementada y no conseguia compilar nada. La version 2.4.3 compilaba
bien pero solo cosas peque�as. Cuando la cancion era grande o el Music
tenia forma de arbol degenerado (es decir, de una lista) que era lo que
sucedia cuando pasabamos un MIDI a Music con las funciones que proporciona
Haskore (loadMidiFile y readMidi) entonces daba un error de pila. Suponemos
que es porque la conversion generaba muchas llaves \{ \} que producirian
muchas llamadas recursibas en lilypond.

\section{Posibles ampliaciones}
Una ampliacion muy interesante habria sido introducir los cifrados que
genera Genaro en la propia partitura y no solo las notas. Lilypond puede
hacerlo aunque habria que hacer la conversion.\\
Ya menos interesante aunque posible habria sido introducir caracteristica 
de articulacion, como puede ser el legato, o cosas como el tempo o 
la intensidad de la nota (seria traducir el velocity a forte, mezoforte, etc.)

\end{document}
















