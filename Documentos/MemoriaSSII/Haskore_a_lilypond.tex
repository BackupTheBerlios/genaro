\documentclass[a4paper]{report}

\usepackage[spanish]{babel}
\usepackage{musixtex}

\title{Haskore a Lilypond}
\author{Roberto Torres de Alba}
\setlength{\hoffset}{-2cm}
\setlength{\voffset}{-2cm}
\setlength{\textwidth}{450pt}

\begin{document}

\tableofcontents

\chapter{Haskore a Lilypond}

\section{Introducci\'on}

%hablar del cygwin
%hablar de sostenidos y bemoles
Este m\'odulo pretende ser un traductor que pasa del arbol de constructoras que representa 
el tipo \emph{Music} a un \emph{String} que sea entendido por Lilypond.
Nosotros hemos usado la version 2.4.3 de Liypond para compilar.\\
\indent En primer lugar ibamos a usar el programa \emph{midi2ly} que viene junto con 
Lilypond y que pasa un archivo MIDI a formato Lilypond. Las pruebas que hicimos
con ellos nos desilusionaron bastante ya que el resultado obtenido estaba lejos
de ser correcto incluso con ejemplos simples 
(hechos con Haskore, es decir, que no estaba la mano de ning\'un humano
que pod\'\i a variar el tiempo en mil\'esimas).\\
\indent Es por ello que intentamos hacer la traducci\'on nosotros mismos a algun
formato que se pudiera compilar a pdf. A diferencia de otros lenguajes
como musicTex o \musixtex, el formato de Lilypond era muy parecido a la
estructura del tipo Music de Haskore. Esto nos motiv\'o a hacer la conversi\'on 
de esta forma.

\section{Traducci\'on}
En esta secci\'on no vamos a entrar en c\'omo es el proceso de conversi\'on
completo ni en todas las caracter\'\i sticas de Lilypond porque, m\'as que nada,
ni yo mismo las conozco.\\
\indent Ya que el tipo Music no posee informaci\'on sobre la obra completa hemos
tenido que hacer un tipo que lo englobe. Ese tipo se llama \emph{CancionLy}
y es muy simple de entender:

\small
\begin{verbatim}
type Armadura = (PitchClass, Modo)
data Modo = Mayor | Menor
     deriving (Eq, Ord, Show, Read, Enum)
type Ritmo = (Int  -- Numero de notas del compas
             ,Int  -- Resolucion del compas
             )
type Instrumento = String
data Clave = Sol   -- Clave de 'Sol'
             | Fa  -- Clave de 'Fa'
             | Bateria -- Clave de 'Bateria'
type Score = (Music, Armadura, Ritmo, Instrumento, Clave)
type Titulo = String
type Compositor = String

type CancionLy = (Titulo, Compositor, [Score])
\end{verbatim}
\normalsize

Muchas de las cosas anteriores se comprenden por lo que no vamos a entrar
a detallarlas. La parte m\'as importante es la traducci\'on de CancionLy
es la parte del Score (pentagrama en ingl\'es) que contiene al tipo Music.

\subsection{Traducci\'on del tipo Music}
Como hemos dicho, Lilypond y Music tienen varias cosas en com\'un que
hacen atractiva su conversi\'on. En concreto son dos: ambos poseen
un operador para secuenciar m\'usica y para paralelizarla.\\
\indent El operador de m\'usica paralela (que se interpreta a la vez) es el :=:
en Haskore. En Lilypond es todo aquello que va entre $<<$ y $>>$. El 
operador secuencia en Haskore es el :+: y en Lilypond es todo aquello
que no est\'a entre $<<$ y $>>$. Lilypond tambi\'en es capaz de agrupar
m\'usica poniendo llaves \{ \}, a lo que llama secuencia.

\small
\begin{verbatim}
deMusicALy :: Music -> String
deMusicALy (Note p dur _) = imprimeNota p dur
deMusicALy (Rest dur)     = imprimeSilencio dur
--   Cambia el operador secuencia de Haskore por el secuencia de Lilypond
deMusicALy (m1 :+: m2)    = " { " ++ deMusicALy m1 ++ " " ++ deMusicALy m2 ++ " } "
--   Cambia el operador paralelo de Haskore por el paralelo de Lilypond
deMusicALy (m1 :=: m2)    = " << " ++ deMusicALy m1 ++ " " ++ deMusicALy m2 ++ " >> " 
deMusicALy (Tempo d m)    = "\\times " ++ imprimeRatio d ++ " { " ++ deMusicALy m ++ " } "
--   Elimina la constructora 'Trans'
deMusicALy (Trans t m)    = deMusicALy (suma t m)
--   El resto de las constructoras son ignoradas
deMusicALy (Instr _ m)    = deMusicALy m
deMusicALy (Player _ m)   = deMusicALy m
deMusicALy (Phrase _ m)   = deMusicALy m
\end{verbatim}
\normalsize

Esta es la funci\'on central del m\'odulo \emph{HaskoreALilypond.hs}. Algunas constructoras
de Music tiene que ser ignoradas porque o no corresponden con nada de
Lilypond o la traducci\'on ser\'\i a tan complicada que llevar\'\i a mucho tiempo y no 
merecer\'\i a la pena.\\

\subsubsection{Traducci\'on de alteraciones}
Cuando hablamos de alteraciones nos referimos a los sostenidos y bemoles musicales.
Como el sistema del tipo Music es atemperado no importa si se ejecuta \# C o bC ya
que son el mismo sonido. Un m\'usico de verdad no usar\'\i a ambos s\'\i mbolos indistintamente
porque no tienen el mismo significado. El problema es que no sabemos c\'omo se usa
un s\'\i mbolo u otro por lo que optamos por una soluci\'on simple y elegante: si
la armadura tiene sostenidos entonces que sean todas las alteraciones sostenidos, e
\emph{idem} para bemoles.\\
\indent La partitura final puede que no sea correcta musicalmente hablando pero, al menos,
es m\'as f\'acil de leer. La funci\'on que lo lleva a cabo es \emph{arreglaAlteraciones}
que no copio aqu\'\i~por ser demasiado extensa.

\subsubsection{Traducci\'on de duraciones}
Por \'ultimo hay que se~nalar un apunte sobre la traducci\'on de duraciones de
notas o silencios. Lilypond s\'olo entiende duraciones que son potencias de
dos, es decir, 1 es la redonda, 2 la blanca, etc. Pero las duraciones en
Haskore son fracciones de enteros. Por ello se ha hecho lo siguiente:

\begin{enumerate}
\item Si el numerador es mayor que uno entonces se repite tantas veces como diga
dicho numerador y todas las notas se ligan.
\item Si el denominador no es potencia de 2 se cambia a la potencia de dos m\'as
cercana. Este \'ultimo paso elimina informacion del tipo Music pero no se qu\'e
podr\'\i amos hacer si llega algo del estilo de 1/27. En caso de dejarselo 
a lilypond este siempre pone una redonda cuando no lo entiende, cosa que
es todav\'\i a peor.
\end{enumerate}

Esto ha motivado la siguiente funcion:

\small
\begin{verbatim}
imprimeNota :: Pitch -> Dur -> String
imprimeNota p dur 
   -- Ponemos puntillo
   | numerador == 3 && denominador > 1 = imprimePitch p ++ 
                                         show (quot denominador 2) ++ "."
   | otherwise                         = eliminaUltimos 2 (concat [imprimePitch p ++ 
                                         show (redondeaAPotenciaDos(denominator dur)) ++ 
                                         "~ " | i <- [1..numerador] ] )
   where numerador = numerator dur;
         denominador = denominator dur;
\end{verbatim}
\normalsize

\section{Problemas}
El principal problema no ha sido la conversi\'on al formato Lilypond. Sabemos
que muchas veces, sobre todo en melod\'\i a y bajo, existe mucha probabilidad de que el denominador de las notas
no sea potencia de dos. Nos hubieramos conformado con que tuvieramos
una ligera aproximaci\'on a las notas que genera Genaro.\\
\indent El principal problema es el propio Lilypond. La version 2.4.2 estaba mal 
implementada y no consegu\'\i a compilar nada. La version 2.4.3 compilaba
bien pero s\'olo cosas peque~nas. Cuando la canci\'on era grande o el Music
ten\'\i a forma de \'arbol degenerado (es decir, de una lista) que era lo que
sucedia cuando pasabamos un MIDI a Music con las funciones que proporciona
Haskore (\emph{loadMidiFile} y \emph{readMidi}) entonces daba un error de pila. Suponemos
que es porque la conversi\'on generaba muchas llaves \{ \} que produc\'\i an
muchas llamadas recursivas en Lilypond.

\section{Posibles ampliaciones}
Una ampliaci\'on muy interesante habr\'\i a sido introducir los cifrados que
genera Genaro en la propia partitura y no s\'olo las notas. Lilypond puede
hacerlo aunque habr\'\i a que hacer la conversi\'on e introducir m\'as informaci\'on en CancionLy.\\
\indent Ya menos interesante, aunque posible, habr\'\i a sido introducir caracter\'\i stica 
de articulaci\'on, como puede ser el \emph{legato}, o cosas como el \emph{tempo} o 
la intensidad de la nota (seria traducir el \emph{velocity} a \emph{forte}, \emph{mezoforte}, etc.)

\end{document}
















