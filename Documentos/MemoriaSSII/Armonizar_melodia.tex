\chapter{Armonizaci\'on de Melod\'\i a}

\section{Introducci\'on}
Armonizar una melod\'\i a consiste, musicalmente hablando, en buscar
acordes que ''vayan bien'' a una cierta melod\'\i a dada. Como
veremos, la frase ''vayan bien'' simplemente es que cada nota
de la melod\'\i a que vamos a armonizar sea parte de las notas que forman el
acorde.\\
Por tanto, la entrada de este m\'odulo es una melod\'\i a (puesta en forma
de \emph{Music} secuencial) y la salida es una progresi\'on que lo armoniza.
Obviamente el usuario va a poder seleccionar diferentes criterios para
dicho fin.\\
\indent El algoritmo completo se compone de dos partes. En una primera parte
buscamos aquellas notas que vamos a querer que sean armonizadas. A esas
notas las vamos a llamar \emph{notas principales} y son aquellas que se pueden identificar
m\'as en un contexto vertical. En la segunda 
vamos a dar un acorde a una o varias notas principales. La concatenaci\'on
de todos los acordes nos van saliendo nos proporciona la progresi\'on de salida de este m\'odulo.\\
\indent Antes de cualquier etapa hay que comprobar que el tipo \emph{Music} que nos
pasan es realmente una melod\'\i a, es decir, un \emph{Music} exclusivamente 
secuencial (que s\'olo posea los operadores \emph{:+:}, \emph{Note} y \emph{Rest}) y que adem\'as
lo transforme en una estructura m\'as manejable:
\small
\begin{verbatim}
type Melodia = [ HaskoreSimple ]
data HaskoreSimple = Nota Pitch Dur
                   | Silencio Dur
     deriving(Eq, Ord, Show, Read)

deHaskoreSecuencialAMelodia :: Music -> Melodia
deHaskoreSecuencialAMelodia (Note p d _) = [ Nota p d ]
deHaskoreSecuencialAMelodia (Rest d)     = [ Silencio d ]
deHaskoreSecuencialAMelodia (m1:+:m2)    = deHaskoreSecuencialAMelodia m1 ++ 
                                                deHaskoreSecuencialAMelodia m2
deHaskoreSecuencialAMelodia _            = error "No es un Music solo secuencial"
\end{verbatim}
\normalsize

\section{B\'usqueda de notas principales}
Una vez que tenemos transformado la melod\'\i a (un \emph{Music} secuencial)
a una lista de notas y silenciones (tipo \emph{Melodia}) tenemos que
seleccionar exclusivamente las notas que vamos a armonizar
(notas principales) asegur\'andonos de que la suma de las
duraciones de las notas principales sea la misma que la de la
melod\'\i a pasada (para que los acordes duren lo mismo que la
melod\'\i a).\\
\indent ?`Qu\'e hacer con las notas que no son principales? Tanto las notas que 
no son principales como los silencios tiene que ser borrados por no 
considerarse parte de la armon\'\i a, sin embargo queremos
que su duraci\'on no se pierda para que la longitud temporal de 
la melod\'\i a sea la misma que la progresi\'on de acordes. Por consiguiente tienen
que ceder su duraci\'on a la nota principal que tengan m\'as a la izquierda.\\
\indent Existe un caso especial: si la primera nota es no principal (es secundaria)
no tiene ninguna nota principal a su izquierda. En ese caso cedera 
su duraci\'on a la que est\'a a su derecha.\\
Como con todo en esta vida existen varias formas de hacerlo. Nosotros
hemos seleccionado dos:

\subsection{Notas de larga duraci\'on}
Consideramos que se van a identificar m\'as en un contexto arm\'onico 
aquellas notas que tengan un duraci\'on larga. Pero, sin embargo,
no podemos decir c\'omo de larga tiene que ser esa duraci\'on, sino que depender\'a de la
canci\'on y, por tanto, ser\'a una apreciaci\'on subjetiva. Es por ello
que tiene que ser pasada como par\'ametro. Por tanto, llegamos a lo siguiente: todas aquellas notas
cuya duraci\'on sea mayor o igual a la duraci\'on m\'\i nima
pasada como par\'ametro ser\'an consideradas como notas largas. Lo 
habitual es que dicha duraci\'on m\'\i nima sea 1/4, que corresponde
a una negra, aunque pueden darse otros casos.

\small
\begin{verbatim}
deMelodiaANotasPrincipales1 :: DurMin -> Melodia -> [NotaPrincipal]
deMelodiaANotasPrincipales1 durMin melodia = reverse (drop 1 (deMelodiaANotasPrincipales1Rec 
                                             durMin (C,0%1) (0%1) (reverse melodia) ))

deMelodiaANotasPrincipales1Rec :: DurMin -> NotaPrincipal -> Dur -> Melodia -> [NotaPrincipal]
deMelodiaANotasPrincipales1Rec durMin (pc, d) durAcumulada [] = [ (pc, d + durAcumulada) ]
deMelodiaANotasPrincipales1Rec durMin ultNotaP durAcumulada ( (Silencio d) : resto ) = 
       deMelodiaANotasPrincipales1Rec durMin ultNotaP (durAcumulada + d) resto
deMelodiaANotasPrincipales1Rec durMin ultNotaP durAcumulada ( (Nota (pc, o) d) : resto ) 
       | d >= durMin  =  ultNotaP : deMelodiaANotasPrincipales1Rec durMin 
                         (pc, d + durAcumulada) (0%1) resto
       | d <  durMin  =  deMelodiaANotasPrincipales1Rec durMin ultNotaP (d + durAcumulada) 
                         resto
\end{verbatim}
\normalsize

\subsection{Un m\'etodo m\'as depurado}
Este m\'etodo es un poco m\'as depurado que el anterior y selecciona a m\'as
notas principales. Tambi\'en tiene en cuenta cosas como el ritmo o 
hacia d\'onde se mueve una nota. Se basa en el criterio que expone
Enric Herrera en su libro Teor\'\i a Musical y Armon\'\i a Moderna 
Volument 1 p\'agina 72. Paso a explicar los diferentes casos:
\begin{enumerate}
\item Nota de larga duraci\'on. Igual que en la secci\'on anterior
\item Nota seguida de silencio de, al menos, su misma duraci\'on.
\item Nota seguida de otra entre las que existe un salto. Suponemos como
salto la distancia en altura de m\'as (estricto) de 2 semitonos.
\item Nota corta que resuelve en su inmediata inferior de tiempo fuerte a d\'ebil
\end{enumerate}

Para saber si el tiempo en el que cae una nota es fuerte o d\'ebil hacemos
lo siguiente. Necesitamos la duraci\'on de toda la canci\'on anterior a esa nota,
incluida la nota, y el denominador de la duraci\'on de dicha 
nota (a lo que hemos llamado \emph{resoluci\'on}). Cambiamos la duraci\'on
por una fracci\'on equivalente que tenga de denominador la resoluci\'on. 
Si la nueva fracci\'on tiene un numerador par entonces el tiempo es fuerte,
si no es d\'ebil.\\
\small
\begin{verbatim}
esTiempoFuerte :: Dur -> Resolucion -> Bool
esTiempoFuerte du r = even num
       where (num, den) = cambiaResolucion du r

cambiaResolucion :: Dur -> Resolucion -> (Int, Int)
cambiaResolucion du r = (newN , newD)
       where (n, d) = ( numerator du, denominator du);
             newN = n * (div r d);
             newD = r
\end{verbatim}
\normalsize
\indent Este procedimiento est\'a pensado y probado para duraciones y resoluciones
que sean potencia de dos, tal y como est\'an preparadas las notas en m\'usica
(es decir, 1/2, 1/4, 1/8, etc.). Es dif\'\i cil conocer el acento del tiempo
si no cumple las condicines ahora dichas y no me comprometo a que sea
correcto el resultado en caso de que no las cumpla.\\
\indent La funci\'on completa es la siguiente:\\

\small

\begin{verbatim}
deMelodiaANotasPrincipales3 :: DurMin -> Melodia -> [NotaPrincipal]
deMelodiaANotasPrincipales3 durMin melodia = reverse (drop 1 (deMelodiaANotasPrincipales3Rec 
       durMin (C,0%1) (0%1) (reverse (catalogaNotasPrincipales3 (0%1) durMin melodia)) ))

deMelodiaANotasPrincipales3Rec :: DurMin -> NotaPrincipal -> Dur -> [(HaskoreSimple, Bool)] 
                                  -> [NotaPrincipal]
deMelodiaANotasPrincipales3Rec durMin (pc, d) durAcumulada [] = [ (pc, d + durAcumulada) ]
deMelodiaANotasPrincipales3Rec durMin ultNotaP durAcumulada ( (Silencio d, _) : resto ) = 
        deMelodiaANotasPrincipales3Rec durMin ultNotaP (durAcumulada + d) resto
deMelodiaANotasPrincipales3Rec durMin ultNotaP durAcumulada ( (Nota (pc, o) d, b) : resto ) 
       | b == True  =  ultNotaP : deMelodiaANotasPrincipales3Rec durMin (pc, d + durAcumulada) 
                                  (0%1) resto
       | b == False =  deMelodiaANotasPrincipales3Rec durMin ultNotaP (d + durAcumulada) resto

type DurAnt = Dur 
catalogaNotasPrincipales3 :: DurAnt -> DurMin -> Melodia -> [(HaskoreSimple, Bool)]
catalogaNotasPrincipales3 _ _ [] = []
catalogaNotasPrincipales3 durAnt durMin (Nota (pc, o) d : resto )
       | d >= durMin = ( Nota (pc, o) d, True ) : catalogaNotasPrincipales3 (durAnt+d) durMin 
                       resto
catalogaNotasPrincipales3 durAnt durMin (Nota (pc, o) d1 : Silencio d2 : resto) 
       | d2 >= d1  = ( Nota (pc, o) d1, True ) : ( Silencio d2 , False) : 
                     catalogaNotasPrincipales3 (durAnt+d1+d2) durMin resto
catalogaNotasPrincipales3 durAnt durMin (Nota (pc1, o1) d1 : Nota (pc2, o2) d2 : resto)
       | abs (pitchClass pc1 - pitchClass pc2) > 2 = ( Nota (pc1, o1) d1, True ) : 
                     catalogaNotasPrincipales3 (durAnt+d1) durMin (Nota (pc2, o2) d2 : resto)
catalogaNotasPrincipales3 durAnt durMin (Nota (pc1, o1) d1 : Nota (pc2, o2) d2 : resto)
       | esTiempoFuerte (durAnt + d1) (denominator d1) && ((dist == 1) || (dist == 2)) = 
                    ( Nota (pc1, o1) d1, True ) : catalogaNotasPrincipales3 (durAnt+d1) 
                     durMin (Nota (pc2, o2) d2 : resto)
       where dist = (pitchClass pc1 - pitchClass pc2)
catalogaNotasPrincipales3 durAnt durMin (Nota (pc, o) d : resto ) = 
        (Nota (pc, o) d , False) : catalogaNotasPrincipales3 (durAnt+d) durMin resto
catalogaNotasPrincipales3 durAnt durMin (Silencio d : resto) = 
        (Silencio d, False) : catalogaNotasPrincipales3 (durAnt+d) durMin resto
\end{verbatim}

\normalsize
\indent La funci\'on \emph{catalogaNotasPricipales} tiene la finalidad de
asignar un booleano a cada nota indicando si es principal o no.
Esto simplemente tiene la finalidad de hacer el algoritmo un 
poco m\'as simple.\\

\section{Armonizaci\'on de notas principales}
Una vez que ya tenemos las notas principales hay que buscar aquellos
acordes que posean a esas notas principales. Cuando hablamos de
poseer a las notas principales nos referimos exclusivamente
a su \emph{PitchClass}.\\
\indent Hay que tener en cuenta una cosa importante. Suponemos que la 
melod\'\i a pasado est\'a en la tonalidad de C Mayor, lo que
significa que las notas C, D, E, F, G, 
A y B son diat\'onicas a la tonalidad
y que el resto son notas crom\'aticas. Esto implica que las notas 
diat\'onicas van a ser armonizadas con acordes diat\'onicos mientras 
que el resto ser\'an armonizadas con acordes crom\'aticos (dominantes
secundarios, disminuidos, etc.).\\
\indent Al igual que antes vamos a tener dos formas de armonizar:

\subsection{Un acorde por nota principal}
Es el caso m\'as simple y musicalmente suena bien cuando las
notas principales son relativamente largas.\\
\indent Como el propio nombre indica a cada nota principal le va a 
asignar un unico acorde por lo que el procedimiento es bastante simple.

\begin{enumerate}
\item Para cada nota principal busca todos los acordes que posean a dicha
noa (en funcion de si es diatonica o no).
\item De todos los acordes que la pueden armonizar se queda con
uno aleatoriamente.
\end{enumerate}
\small
\begin{verbatim}
armonizaNotasPrincipales1 :: RandomGen g => g -> ModoAcordes 
                             -> [NotaPrincipal] -> Progresion
armonizaNotasPrincipales1 _   _  [] = []
armonizaNotasPrincipales1 gen ma (notaP : resto) = cifradoYDur : 
     armonizaNotasPrincipales1 sigGen ma resto
           where (cifradoYDur, sigGen) = armonizaNotaPrincipal1 gen ma notaP

armonizaNotaPrincipal1 :: RandomGen g => g -> ModoAcordes -> NotaPrincipal 
                          -> ((Cifrado, Dur), g)
armonizaNotaPrincipal1 gen ma (notaP, dur) = ((cifradoAleatorio, dur), sigGen)
           where cifradosCandidatos = buscaCifradosCandidatosDeCMayor ma notaP
                 (cifradoAleatorio , sigGen )= elementoAleatorio gen 
                                               cifradosCandidatos 
\end{verbatim}
\normalsize

\subsection{M\'as largo posible}
La idea de este m\'etodo es tambi\'en bastante sencilla. Consiste en 
buscar el acorde que englobe el mayor numero de notas principales
posible. Existe la posibilidad de introducir una duraci\'on m\'axima
para ese acorde y as\'\i ~evitar que generen acordes demasiado
largos aunque si buscamos un \'unico acorde para armonizar
toda una melod\'\i a (que puede no conseguirse) basta con poner esa
duraci\'on muy grande.\\
El algoritmo es como sigue:
\begin{enumerate}
\item Busca todos los acordes para armonizar una nota y continua
con el resto de notas.
\item Para el resto vuelve a buscar los acordes candidatos y 
realiza la interseccion con la lista del punto 1. De esa
forma nos vamos quedando con los acordes que armonizan las
notas anteriores.
\item Continuamos asi hasta que dicha lista sea vacia o la duracion
del acorde exceda la duracion maxima. En ese caso elegimos un 
acorde al azar (de la lista anterior porque la actual esta vacia, claro)
y volvemos al punto 1.
\end{enumerate}

\small
\begin{verbatim}
type CifradosCandidatos = [Cifrado]
type DurAcordeAcumulado = Dur

-- Para comenzar con los cifrados condidatos
armonizaNotasPrincipales3 :: RandomGen g => g -> ModoAcordes -> DurMaxA 
                             -> [NotaPrincipal] -> Progresion
armonizaNotasPrincipales3 _  _ durMaxA [] = []
armonizaNotasPrincipales3 gen ma durMaxA ( (pc, d) : resto)
     | d == durMaxA  =  (cifradoAleatorio, d ) : armonizaNotasPrincipales3 
                        newGen ma durMaxA resto   
     | d >  durMaxA  =  (cifradoAleatorio, durMaxA ) : armonizaNotasPrincipales3 
                        newGen ma durMaxA ( (pc, d - durMaxA) : resto)
     | d <  durMaxA  =  armonizaNotasPrincipales3' gen ma durMaxA cifradosCandidatos 
                        d resto  
     where cifradosCandidatos = buscaCifradosCandidatosDeCMayor ma pc;
           (cifradoAleatorio, newGen) = elementoAleatorio gen cifradosCandidatos 

-- Cuando ya tenemos una lista de cifrados candidatos
armonizaNotasPrincipales3' :: RandomGen g => g -> ModoAcordes -> DurMaxA -> 
                              CifradosCandidatos -> DurAcordeAcumulado -> [NotaPrincipal] 
                               -> Progresion
armonizaNotasPrincipales3' gen ma durMaxA cifCan durA [] = [(cifAlea, durA)]
    where (cifAlea, newGen) = elementoAleatorio gen cifCan 
armonizaNotasPrincipales3' gen ma durMaxA cifCan durA ( (pc, d) : resto) 
    | durA == durMaxA                        = (cifAlea , durMaxA) : 
            armonizaNotasPrincipales3 newGen ma durMaxA ( (pc, d) : resto )
    | durA + d == durMaxA && newCifCan /= [] = ( newCifAlea , d + durA) : 
            armonizaNotasPrincipales3 newGen2 ma durMaxA resto        
    | durA + d == durMaxA && newCifCan == [] = ( cifAlea , durA) : 
            armonizaNotasPrincipales3 newGen ma durMaxA ( (pc, d) : resto)   
    | durA + d <  durMaxA && newCifCan /= [] = 
            armonizaNotasPrincipales3' gen ma durMaxA newCifCan (durA + d) resto                 
    | durA + d <  durMaxA && newCifCan == [] = ( cifAlea , durA) : 
            armonizaNotasPrincipales3 newGen ma durMaxA ( (pc, d) : resto)   
    | durA + d >  durMaxA && newCifCan /= [] = ( newCifAlea , durMaxA) : 
            armonizaNotasPrincipales3 newGen2 ma durMaxA ( (pc, d - durMaxA + durA) : resto)
    | durA + d >  durMaxA && newCifCan == [] = ( cifAlea , durA) : 
            armonizaNotasPrincipales3 newGen ma durMaxA ( (pc, d) : resto)   
    where newCifCan = eliminaCifradosNoValidos cifCan (buscaCifradosCandidatosDeCMayor ma pc);
          (cifAlea, newGen) = elementoAleatorio gen cifCan;
          (newCifAlea, newGen2) = elementoAleatorio gen newCifCan;
\end{verbatim}

\normalsize

\section{Salida del m\'odulo}
La salida de este m\'odulo es una progresi\'on de acordes como ya hemos visto.
En Genaro consiste en una lista formada por parejas de cifrado m\'as
duraci\'on. El cifrado indica la especie del acorde (ej, D m, Re menor)
y la duraci\'on el tiempo que se extiende. Prolog tambi\'en usa las mismas
progresiones pero con otro formato ya que Prolog no entiende de tipos,
solamente conoce t\'erminos (funciones y variables).\\
\indent El interfaz gr\'afico
est\'a implementado en C++ y en un principio se pens\'o en que parseara
progresiones en formato Prolog. Ahora que tenemos progresiones en formato
Haskell podemos hacer dos cosas: que Haskell escriba la progresion haciendo
un \emph{show} y que se construya otro parser en el interfaz o usar el mismo
parser pero escribir la progresi\'on en formato Prolog. Se opt\'o por hacerlo
de esta \'ultima manera porque era la m\'as f\'acil.\\
\indent Por eso cuando se necesita que Haskell escriba en un fichero la progresi\'on
que ha generado el armonizador antes tiene que llamar a la funci\'on 
\emph{deProgresionAString} y \emph{escribeProgresionComoProlog} que se encuentran
en el m\'odulo \emph{Progresiones.hs}.
