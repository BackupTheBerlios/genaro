\documentclass[a4paper,12pt]{article}
\usepackage[spanish,activeacute]{babel}
%en author tengo que poner los acentos asi pq el activeacute aún no funciona en author
\author{Juan Rodr\'iguez Hortal\'a}
\title{Generador de lineas de bajo}
\frenchspacing

\begin{document}
\maketitle
\tableofcontents
\section{Introducci'on}
El m'odulo generador de melod'ias se encarga de para componer melod'ias a partir de progresiones de acordes y un patr'on r'itmico. Se utilizan las reglas de la armon'ia para que la melod'ia resultado este relacionada con la progresi'on y `quede bien` al sonar a la vez que la traducci'on a m'usica de 'esta.
\newline
Todas las melod'ias compuestas por Genaro son monof'onicas, esto es, son sucesiones de notas en el tiempo tales que no suena m'as de una nota a la vez. Las notas no se superponen en el tiempo como en el caso de los acompa~namientos. Esta simplificaci'on facilita mucho el proceso de composici'on.

\section{Abstracciones empleadas}
La generaci'on de melod'ias de Genaro se basa en dos abstracciones, una \emph{curva mel'odica} y una \emph{lista de acentos}:
        \begin{enumerate}
        \item La curva mel'odica: es una lista de enteros que se entienden como saltos dentro de una escala que no se ha fijado todav'ia.
                \begin{itemize}
                \item Un salto de 0 indica quedarse en la misma nota, un salto de n con n distinto de 0 indica moverse n notas dentro de una escala, hacia arriba en el caso de que n sea positivo y hacia abajo si es negativo.
                \item Una curva mel'odica se refiere por tanto a las alturas de las notas solamente, no indica nada referido a duraciones ni al ritmo.
                \item Una curva melod'ica representa las relaciones de altura relativa entre las notas que conforman la melod'ia, y en ning'un momento se refiere a alturas absolutas. En un paso determinado del proceso de generaci'on de melod'ias se realiza la correspondencia entre una curva mel'odica y una lista de alturas, especificando una escala y una altura de partida que fija una referencia absoluta desde la que comenzar el proceso.
                \item Al pasar a la lista de alturas es posible que la altura de partida no pertenezca a la escala. En ese caso se entender'a que en el primer punto de la curva saltamos a una altura de la escala.
                \end{itemize}
        \item La lista de acentos: es una lista de parejas (float, fraccion entera) que expresa un perfil r'itmico de la melod'ia.
                \begin{itemize}
                \item Cada elemento de la lista de acentos representa la duraci'on de una posible nota (segunda componente) y la fuerza con la que 'esta se atacar'ia por parte del instrumentista que interpretara la melod'ia (primera componente):
                        \begin{enumerate}
                        \item[(a)] Las duraciones se miden como figuras musicales.
                        \item[(b)] La intensidad con que se ataca cada nota se mide de 0 a 100. Los casos de intensidades que valen 0 corresponden a los silencios, componentes fundamentales de la m'usica.
                        \end{enumerate}
                \item Las listas de acentos se construyen de forma determinista procesando patrones r'itmicos pero su aplicaci'on no es absoluto determinista, como se ver'a m'as adelante. Se pueden entender como la correspondencia de una dimensi'on de los patrones r'itmicos, que tienen dos dimensiones.
                \end{itemize}
        \end{enumerate}

\section{Algoritmo de generaci'on de melod'ias}
Utiliza las abstracciones anteriores. Para saber obtener una melod'ia para una progresi'on primero debemos aprender a obtener una melod'ia para un solo acorde, y luego a enlazar las melod'ias de cada acorde.
\newline
El algoritmo que hace una melod'ia para un solo acorde recibe de entrada una curva mel'odica, una lista de acentos, el acorde sobre el que tiene que construir la melod'ia, la altura desde la que empezar la melod'ia y una serie de par'ametros enteros cuya utilidad explicaremos despu'es. Partiendo esa informacion aplicamos los siguientes pasos:
        \begin{enumerate}
        \item Ajustar curva mel'odica con lista de acentos: se modifica la lista de acentos para que tenga la misma longitud que la curva aleatoria. Para ello se dividen en dos o se unen parejas de acentos, seg'un la lista de acentos sea m'as corta o m'as larga que la curva mel'odica. Los acento que se dividen o fusionan se eligen aleatoriamente.
        \item Elegir los acentos: se elige de forma aleatoria una sublista de tama~no aleatorio de la lista de acentos, dando mayor probabilidad de ser elegidos a los acentos m'as fuertes. También se guardan en otra lista los acentos rechazados.
        \item Elegir los puntos de la curva mel'odica: se hace la correspondencia entre la curva mel'odica y una lista de alturas de nota empleando la escala correspondiente al acorde y la altura de partida. Estas alturas concretas ya pueden ser valoradas como m'as o menos estables en el contexto arm'onico que establece el acorde. Se elige una sublista aleatoria de esta lista de alturas, de la misma longitud que la sublista de acentos anterior, dando mayor probabilidad de ser elegidas a las alturas m'as estables.
        \item Construir la primera versi'on de la melod'ia: a partir de las tres sublistas anteriores se construyen dos listas, una a partir de los acentos y alturas elegidos, la lista de notas que suenan; y otra a partir de la lista de acentos rechazados, la lista de silencios o notas que no suenan. Estas dos listas se mezclan siguiendo el orden de los acentos en la lista de acentos original, y se obtiene una primera versi'on de la melod'ia.
        \item Mutar la melod'ia: el resto del proceso consiste en aplicar mutaciones sucesivas a la primera versi'on de la melod'ia. Estas mutaciones se aplican siempre en el mismo orden y seg'un indican los 4 enteros de entrada al algortimo.
                \begin{enumerate}
                \item[i)] Alargar notas: considera todas las notas de la melod'ia que no sean silencios y a las que siga inmediatamente un silencio. Elige aleatoriamente un n'umero aleatorio de estas notas y alarga su duraci'on con la del silencio que les segu'ia, elimin'andolo. Se aplica tantas veces como indique el primer entero.
                \item[ii)] Dividir notas: considera todas las notas de la melod'ia que no sean silencios y a las que siga una nota que no sea un silencio, aunque no sea inmediatamente (aunque haya silencios entre estas dos notas no silencio). Elije aleatoriamente una sola de estas notas y elige aleatoriamente una altura entre la de la nota elegida y la de la primera nota no silencio que le sigue en la melodia. Despu'es elimina los silencios entre estas dos notas e en su lugar inserta una nueva nota y silencios de forma que la nueva nota tiene la altura elegida anteriormente y est'a a la misma distancia en el tiempo de las dos notas. En resumen, calcula la distancia en el tiempo entre dos notas e introduce una nota entre otras ellas, en la mitad de la distancia. Se aplica tantas veces como indique el segundo entero.
                \item[iii)] Dividir notas fino: es como dividir notas solo que en vez de dividir por dos divide por una potencia de dos. Se aplica tantas veces como indique el tercer entero, y en cada aplicaci'on divide por una potencia de dos de exponente elegido aleatoriamente entre 0 y el cuarto entero.
                \end{enumerate}
        \end{enumerate}

Una vez ya sabemos construir una melod'ia para un acorde construir una melod'ia para una progersi'on es sencillo. El algoritmo recibe como entrada una progresi'on, un patr'on ritmico y una serie de par'ametros enteros entre los que se incluyen los anteriores. Partiendo esa informacion aplicamos los siguientes pasos:
        \begin{enumerate}
        \item Generar la curva mel'odica: Genaro tiene un algoritmo de generaci'on aleatoria de curvas mel'odicas. Se le especifica el n'umero de puntos deseado, el salto m'aximo (valor absoluto m'aximo de un punto de la curva) y un peso para el cambio de direcci'on de la curva (la curva cambia de direcci'on cuando pasa de hacer saltos positivos a hacer saltos negativos o vicecersa). El algoritmo genera una curva que cumple esas restricciones, y para ello parte de un primer punto que vale 0, y de una direcci'on de movimiento (arriba o abajo) elegida aleatoriamente. Para el resto de puntos elige aleatoriamente si cambia de direcci'on o no teniendo en cuenta el peso suministrado y el tiempo que lleva `caminando` en una direcci'on, cuanto m'as rato lleve mayor es la probabilidad de que cambie de direcci'on con un salto mayor. Si no cambia de direcci'on son m'as probables los saltos peque~nos. Seg'un estas probabilidades e intervalos de valores v'alidos va generando aleatoriamente los puntos de la curva.
        \item Distribuir la curva entre los acordes: La curva mel'odica se genera para toda la progresi'on por lo que debe distribuirse entre los acordes de la progresi'on. A cada acorde le corresponde una cantidad de puntos de la curva seg'un su duraci'on, dando m'as puntos a los acordes m'as largos. A cada acorde le corresponde un porcentaje del n'umero de puntos de la curva igual a su duraci'on entre la duraci'on total de la progresi'on. Como esos porcentajes multiplicados por el n'umero de puntos de la curva no suelen dar lugar a enteros se distribuye la parte entera y lo que sobra se reparte resolviendo aleatoriamente los conflictos. Una vez decidido cuantos puntos corresponden a cada acorde estos se reparten en orden, es decir, que si a cada acorde i le corresponden $n_{i}$ puntos, entonces al acorde 1 le corresponden los puntos del 1 al $n_{i}$, al acorde 2 del $n_{i}$ + 1 al 2 * ($n_{i}$), etc...

        \end{enumerate}


\section{Implementaci'on}
El m'odulo generador de melod'ias se apoya en la librer'ia Haskore para Haskell para componer melod'ias sobre un acompa~namiento.

contar lo del parser y lo de progresiones q es como en prolog pero traducido

\section {Capacidad de ampliaci'on}

\end{document}





