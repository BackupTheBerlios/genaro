\chapter{Generador de progresiones de acordes}
\section{Introducci'on}
El m'odulo generador de acordes se encarga de generar y modificar \emph{progresiones de acordes}. Para comprender como funciona y para que sirve primero debemos entender el concepto de progresi'on de acordes.

\section {Acordes y progresiones de acordes} 

Recordemos que a un nivel m'as alto de abstracci'on un acorde se puede entender como una agrupaci'on de alturas musicales que se considera que tienen un significado conjunto. Ese ser'a el grado de abstracci'on al que trabajaremos en esta fase de la generaci'on de m'usica, y llamaremos \emph{cifrado} a la representaci'on de los acordes a este nivel de abstracci'on. 

Como este concepto de acorde solamente se refiere a la altura de los sonidos, emparejaremos los cifrados con duraciones (figuras), que expresaran el espacio de tiempo que ocupará la serie de notas, es decir, la m'usica, en la que se concretar'a el acorde. Y a las listas de cifrados emparejados con figuras las llamaremos \emph{progresiones de acordes}. Estas listas expresan la sucesi'on de los acordes a lo largo del tiempo. En una fase posterior de la generaci'on de m'usica se realizar'a la citada correspondencia entre parejas (cifrado, figura) y m'usicas, por tanto entre progresiones de acordes y m'usicas (la m'usica que corresponde a una progresi'on es la que resulta de suceder en el tiempo las m'usicas que corresponden a cada par perteneciente a la progresi'on, desde el primero al 'ultimo). Las diversas maneras en que se puede hacer esta correspondencia permitiran generar m'usicas muy diversas a partir de una misma progresi'on de acordes.

\section {Impacto de los acordes en el resto de la m'usica}
Seg'un las reglas de la Armon'ia, los acordes establecen una jerarqu'ia de sonidos, definiendo un \emph{contexto arm'onico} que a su vez implica una funci'on que valora alturas musicales haci'endole corresponder un entero que llamamos estabilidad de esa altura en ese acorde. Esta funci'on se emplear'a para elegir las notas del bajo y la melod'ia que se compongan a partir de un acompa~namiento, teniendo mayor probabilidad de aparecer en una melod'ia o en un bajo las notas m'as estables.

El contexto arm'onico de un acorde esta determiando no s'olo por el acorde sino tambi'en por su relaci'on con un eje o contexto global que llamamos \emph{tonalidad}, y que define la estabilidad del acorde dentro de este contexto global, y cu'ales son los acordes que es m'as probable que le sigan en una progresi'on. Así que las progresiones de acordes siguen una cierta l'ogica, estando determinado en cierto medida un acorde por los acordes anteriores a 'el y por los acordes que le siguen en la progresi'on. Esto es debido a que, seg'un el sistema de armon'ia empleado en Genaro, la m'usica es un juego entre estabilidad e inestabilidad, los acordes inestables generan un movimiento o \emph{tensi'on} que termina por liberarse o \emph{resolverse} sobre acordes m'as estables. La forma en que Genaro simula esto se explica en la secci'on siguiente.

\section {Algoritmo de generaci'on de progresiones de acordes}
El algoritmo consiste en partir de una progresi'on inicial, que llamamos \emph{progresi'on semilla}, ajustarla para que tenga el n'umero de compases especificado como entrada del algoritmo, y aplicarle varias transformaciones aleatorias (\emph{mutaciones}) hasta conseguir la progresion resultado. Los pasos a seguir son los siguientes:

\begin{enumerate}
\item Elegir una progresi'on semilla: 
        \begin{itemize}
        \item Se elige de entre una base de datos de progresiones t'ipicas, por lo que esta semilla no se genera aleatoriamente, aunque si que se elige aleatoriamente.
        \item Las progresiones semillas que hay actualmente en la base de datos duran entre 1 y 4 compases.
        \item La aplicaci'on de una cantidad suficiente de mutaciones garantiza que las progresiones genaradas sean variadas.
        \end{itemize}       
\item Ajustar el n'umero de compases:
        \begin{itemize}
        \item Para ajustar el n'umero de compases de la progresi'on semilla se restringe las progresiones candidatas a ser semilla a aquellas cuyo duraci'on en n'umero de compases sea divisor del n'umero de compases que es entrada del algortimo. Luego se multiplica el n'umero de compases de todos los acordes de la semilla por la divisi´on del n'umero de compases de entrada entre el n'umero de compases de la semilla, para lograr una progresi'on semilla con el n'umero de compases deseado.
        \item Este sistema funciona porque en las progresiones semilla todos los acordes tienen la misma duraci'on. Si tuvieran duraciones distintas se podr'ia modificar f'acilmente el algoritmo para lograr un encaje de duraciones de otra manera.
        \item El hecho de contar con una progresi'on semilla que dura un comp'as garantiza que siempre se pueda lograr una progresi'on semilla ajustada a cualquier n'umero de compases de entrada, ya que el 1 divide a cualquier n'umero.
        \end{itemize}       
\item Mutar la progresi'on: Las mutaciones son transformaciones que respetan dos propiedades que cumplen todas las propiedades semilla. De esta manera mantenemos como invariante que las progresiones con las que trabajamos cumplen estas propiedades, y podemos afirmar que la progresi'on resultado tambi'en las cumple. Las propiedades son:

        \begin{itemize}
        \item Relaci'on con la tonalidad: Todos los acordes de la progresi'on, o bien pertenecen a la tonalidad (en t'ermino t'ecnicos, son diat'onicos), o bien est'an estrechamente relacionados con 'esta (el caso de los dominanetes secundarios y IIm7). 
        \item Respeto del r'itmo arm'onico: Esta es otra cuesti'on t'ecnica que se puede resumir en que los acordes que se consideran m'as estables valor'andolos en el contexto que establece la tonalidad, se sit'uan en tiempos del comp'as que se consideran m'as fuertes.
        \end{itemize}

        Sin entrar en tecnicismos, se trata de modificar las progresiones sin que estas pierdan su musicalidad, ni su sentido al interpretarlas seg'un las reglas de la armon'ia. Adem'as las mutaciones no cambian el n'umero de compases de la progresi'on, por lo que basta con ajustar el n'umero de compases en la semilla para conseguir el n'umero de compases deseado.
\newline
\newline
        \emph{Las mutaciones son sustituciones de un acorde de la progresi'on por una lista de acordes}, y son aleatorias en dos sentidos:
                \begin{enumerate}
                \item[(a)] La mutaci'on a aplicar se elige aleatoriamente
                \item[(b)] El acorde a mutar se elige aleatoriamente
                \end{enumerate}

        Podemos expresar las mutaciones mediante reglas de reescritura, que suponemos indetermin'istas ya que 'este m'odulo aplica las mutaciones aleatoriamente. Las mutaciones disponibles son \emph{cambia\_acordes}, \emph{aniade\_acordes}, \emph{quita\_acordes}, \emph{dominante\_secundario} y \emph{iim7}:

        \begin{verbatim}
cambia_acordes([]) = []
cambia_acordes([Ac| Acs]) = [acorde_equivalente(Ac)| Acs]
cambia_acordes([Ac| Acs]) = [Ac| cambia_acordes(Acs)]

aniade_acordes([]) = []
aniade_acordes([Ac| Acs]) = pareja_dividida(Ac) ++ Acs
aniade_acordes([Ac| Acs]) = [Ac| aniade_acordes(Acs)]

pareja_dividida(Ac) =  
    [pon_duracion(Ac, D/2)
    ,pon_duracion(acorde_equivalente(Ac), D/2)] <== dame_duracion(Ac) == D


quita_acordes([]) = []
quita_acordes([Ac1, Ac2| Acs]) = pareja_unida(Ac1, Ac2) ++ Acs
        <== equivalentes(Ac1, Ac2) == true
quita_acordes([Ac| Acs]) = [Ac| quita_acordes(Acs)]

pareja_unida(Ac1, Ac2) =  [pon_duracion(acorde_equivalente(Ac1),D1+D2)] 
        <== dame_duracion(Ac1) = D1
           ,dame_duracion(Ac2) = D2

dominante_secundario([]) = []
dominante_secundario([Ac1, Ac2| Acs]) = monta_dom_sec(Ac1) ++ [Ac2|Acs]
        <== candidato_dominante(Ac1, Ac2) == true
dominante_secundario([Ac| Acs]) = [Ac| dominante_secundario(Acs)]

monta_dom_sec(Ac) = [pon_duracion(Ac, D/4)
                    ,pon_duracion(acorde_dominante_de(Ac), D/4)
                    ,pon_duracion(Ac, D/2)
                     ]
        <== dame_duracion(Ac) = D

iim7([]) = []
iim7([Ac1, Ac2| Acs]) = monta_iim7(Ac1) ++ [Ac2|Acs]
        <== candidato_iim7(Ac1, Ac2) == true
iim7([Ac| Acs]) = [Ac| iim7(Acs)]

monta_iim7(Ac) = [pon_duracion(acorde_iim7_de(Ac), D/2)
                 ,pon_duracion(Ac, D/2)
                 ]
        <== dame_duracion(Ac) = D
        \end{verbatim}
        
\end{enumerate}




\section {Implementaci'on}
El lenguaje de implementaci'on empleado en esta parte de Genaro fu'e SWI-Prolog. Al principio us'abamos Sicstus Prolog pero terminamos pas'andonos a SWI-Prolog principalmente porque es libre, y porque el algoritmo de generaci'on de progresiones no necesita una gran potencia de proceso para ser implementado, y para implementar este algoritmo es para lo 'unico que usamos Prolog. Adem'as la vuelta atr'as de Prolog se aprovecha para hacer indeterminista el algoritmo.
\newline
Como Prolog no tiene tipos se dise~naron una serie de predicados que definen las estructuras con las que trabajamos en este m'odulo, como por ejemplo las progresiones. Estos predicados tienen 'exito cuando sus argumentos de entrada son t'erminos con la estructura deseada, y no son llamados nunca, se desarrollaron solamente como especificaci'on.
\newline
La representaci'on de progresiones en Prolog es la siguiente:
        \begin{verbatim}
es_progresion(progresion(P)) :- es_listaDeCifrados(P).
es_listaDeCifrados([]).
es_listaDeCifrados([(C, F)|Cs]) :- es_cifrado(C),es_figura(F)
                                  ,es_listaDeCifrados(Cs).

es_cifrado(cifrado(G,M)) :- es_grado(G), es_matricula(M).

es_matricula(matricula(M)) :- 
        member(M, [mayor,m,au,dis,6,m6,m7b5, maj7,7,m7,mMaj7,au7,dis7]).

es_grado(grado(G)) :- es_interSimple(interSimple(G)).
es_grado(grado(v7 / G)) :- es_grado(grado(G)).
es_grado(grado(iim7 / G)) :- es_grado(grado(G)).

es_interSimple(interSimple(G)) :-	
	member(G, [i, bii, ii, biii, iii, iv, bv, v, auv, vi, bvii, vii]).
es_interSimple(interSimple(G)) :-	
 	member(G, [bbii, bbiii, auii, biv, auiii, auiv, bbvi, bvi, auvi, bviii, auviii]).
        \end{verbatim}
En 'este m'odulo trabaja manejando t'erminos con esta estructura. El resultado 'ultimo de la generaci'on de progresiones un archivo de texto de nombre especificado en la entrada de Prolog, que contiene un t'ermino que cumple el predicado es\_progresion/1. Un ejemplo de fichero escrito por Prolog ser'ia (los saltos de linea se han introducido para facilitar la lectura y en realidad no pertenecer'ian al fichero):
        \begin{verbatim}
progresion([ (cifrado(grado(i), matricula(maj7)), figura(2, 4))
           , (cifrado(grado(iii), matricula(m7)), figura(2, 4))
           , (cifrado(grado(vi), matricula(m7)), figura(2, 8))
           , (cifrado(grado(vii), matricula(m7b5)), figura(2, 8))
           , (cifrado(grado(vii), matricula(m7b5)), figura(2, 4))
           , (cifrado(grado(iii), matricula(m7)), figura(2, 4))
          ]).
        \end{verbatim}

\section {Otros usos de 'este m'odulo}
La interfaz gr'afica desarrollada en C++ permite al usuario un acceso c'omodo y f'acil a las diversas funcionalidades que ofrece este m'odulo, sin tener que lidiar con Prolog por consola. As'i pu'es la interacci'on con el generador de acordes es sencilla y puede aumentarse la complejidad de la manipulaci'on de las progresiones sin que esto le resulte complicado al usuario. Las funcionalidades que ofrece de este m'odulo conbinado con el interfaz gr'afico son:
\begin{itemize}
\item Generaci'on de progresiones de acordes: Se especifica el n'umero de compases y el n'umero de mutaciones a aplicar.  Las mutaciones se han dividido en dos grupos, el primero integrado por cambia\_acordes, aniade\_acordes y quita\_acordes, y el segundo por dominante\_secundario y iim7. Se puede ser m'as fino en la especificaci'on de las mutaciones e indicar el n'umero de mutaciones de cada grupo a aplicar, o incluso de cada tipo concreto, para tener mayor control. En todos los casos el orden en que se aplican las mutaciones es aleatorio.
\item Salvado y carga de las progresiones de acordes: Se pueden guardar las progresiones generadas para recuperarlas en futuros proyectos o emplearlas para distintas pistas o bloques de un proyecto. Apilar pistas de acompañamiento con la misma progresi'on de acordes pero con estas progresiones transformadas en m'usica de diferentes maneras,o incluso con distintas fuentes de sonido, es una opción interesante para componer.
\item Modificaci'on de una progresi'on: A una progresi'on tambi'en se le pueden aplicar mutaciones desp'ues de ser creada o cargada en Genaro. Estas mutaciones se pueden especificar con la misma finura que en la genaraci'on e incluso se puede especificar un acorde concreto para mutar. Es una forma de emplear a Genaro como asistente a la composici'on.
\item Edici'on manual de una progresi'on: Genaro incorpora un editor de progresiones de acordes, para modificar una progresi'on generado por Genaro que es interesante pero falla en alg'un punto, para que el usuario introduzca una progresi'on que le gusta y Genaro le haga una melod'ia y un bajo, o para que genaro se encargue de pasar la progresi'on a m'usica concreta... otra vez Genaro como asistente a la composici'on.
\end{itemize}

\section {Capacidad de ampliaci'on}
Este m'odulo es f'acilmente ampliable, en parte por las caracter'isticas de Prolog. No hay clases ni interfaces a los que ce~nirse, basta con ajustarse a la especificaci'on del predicado es\_progresion/1 para trabajar en su mismo idioma. Las vias m'as evidentes por las que se podr'ia ampliar este m'odulo son:
        \begin{enumerate}
        \item Adici'on de nuevas mutaciones: basta con crear predicados que tengan como entrada un t'ermino que cumpla es\_progresion/1 y como salida otro t'ermino que tambi'en lo haga. Las reglas de la armon'ia sugieren la introducci'on de nuevas mutaciones usando t'ecnicas como disminuidos de paso, intercambio modal, dominantes sustitutos. La extensi'on del m'odulo por esta v'ia ser'ia muy sencilla.
        \item Adici'on de nuevas progresiones semilla: bastar'ia con a~niadir a la base de datos nuevas progresiones que cumplieran las dos condiciones antes mencionadas, y que tuvieran inter'es est'etico. Esta extensi'on ser'ia inmediata.
        \item Modulacion y enlace de progresiones: se tratar'ia de construir progresiones pensadas para suceder a otras progresiones, siguiendo una l'ogica marcada por las leyes de la armon'ia. La dificultad aqu'i estar'ia en la cuesti'on te'orica musical m'as que en la inform'atica, puesto que la abstracci'on de las progresiones de acordes y su implementaci'on suponen un acceso sencillo para enfrentarse a este problema.
        \item Algo totalmente diferente: Cualquier programa en cualquier lenguaje que lea y escriba archivos de texto en el formato especificado por el predicado es\_progresion/1 puede comunicarse con 'este m'odulo y extenderlo o sustituirlo.
        \end{enumerate}
