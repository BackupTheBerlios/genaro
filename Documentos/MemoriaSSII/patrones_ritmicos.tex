\documentclass[a4paper]{report}
\usepackage[spanish]{babel}
\usepackage{musixtex}
\title{Patrones Ritmicos}
\author{Roberto Torres de Alba}

\setlength{\hoffset}{-2cm}
\setlength{\voffset}{-2cm}
\setlength{\textwidth}{450pt}

\begin{document}

\tableofcontents

\chapter{Patrones R\'\i tmicos}

\section{Introducci\'on}
La etapa del patr\'on r\'\i tmico es la que se ejecuta posteriormente a la de traducci\'on de los cifrados. Es por ello que
lo que recibe esta etapa es una lista de lo que hemos llamado ''acordes ordenados'' ( el tipo emph{[AcordeOrdenado]} ). La idea principal 
de esta parte es repartir las voces de los acordes ordenados en el tiempo. Es por ello que vamos a buscar una estructura
que sea capaz de organizar las voces en el tiempo, sirva para cualquier acorde independientemente del numero de voces
o duracion que posea y se pueda guardar para posteriores usos.\\
Para ilustar estas ideas veamos un ejemplo simple. Supongamos que de la etapa anterior de traduccion de cifrados nos
llega lo siguiente [ ([(C,4),(E,4),(G,4),(C,4)],1%1) ]. Con estas notas podemos hacer, entre otras cosas, tocarlas todas
a la vez o arpegiarlas:\\

\begin{music}
\instrumentnumber{1}
\setname1{}
\setstaffs11
\startextract
\notes\ibu0f0\zw c\zw e\zw g\wh j\enotes
\bar
\notes\ibu0f0\qa c\qa e\qa g\qa j\enotes
\endextract
\end{music}

Aunque las notas son las mismas ellas estan dispersas en el tiempo de diferente forma.\\

\section{Tipo y Funcionamiento}
El tipo del patron ritmico es el siguiente:\\
\begin{verbatim}
type Voz = Int      -- Voz: la primera voz, en este caso, es la mas grave
type Acento = Float -- Acento: intensidad con que se ejecuta 
                    --         esa nota. Valor de 0 a 100
type Ligado = Bool  -- Ligado: indica si una voz de una columna esta 
                    --         ligada con la se la siguiente columna.
type URV = [(Voz, Acento, Ligado)]	-- Unidad de ritmo vertical, especifica todas las filas de una unica columna
type URH = Dur 				-- Unidad de ritmo horizontal, especifica la duracion de una columna
type UPR = ( URV , URH)             -- Unidad del patron ritmico, especifica completamente toda la informacion necesaria
                                    -- para una columna (con todas sus filas) en la matriz que representa el patron
type AlturaPatron = Int			-- numero maximo de voces que posee el patron
type MatrizRitmica = [UPR]          -- Una lista de columnas, vamos, como una matriz

type PatronRitmico = (AlturaPatron, MatrizRitmica)
\end{verbatim}

La idea mas simple de ver un patron ritmico es como una matriz (aunque no lo sea exactamente ya que el numero de filas
por columna varia) que representa una plantilla con agujeros en el que su altura (numero de filas) es el numero de 
voces y la anchura (numero de columnas) una duracion enforma de fraccion de Haskore. Dicha plantilla se encaja encima 
del acorde ordenado; aquellas voces que caigan dentro de un agujero se ejecutaran en ese instante de tiempo. Por ejemplo, supongamos la
siguiente plantilla con agujeros que representa el arpegio anterio de cuatro voces seria algo asi:

\begin{verbatim}
- - - X
- - X -
- X - -
X - - -
\end{verbatim}

donde las X representan los agujeros. Ahora supongamos que la plantilla es esta:

\begin{verbatim}
- X - -
- - X -
- - X -
X - - X
\end{verbatim}

entonces para el siguiente acorde tendriamos lo siguiente cuando aplicaramos el patron:

\begin{music}
\instrumentnumber{1}
\setname1{}
\setstaffs11
\startextract
\notes\ibu0f0\zw c\zw e\zw g\wh j\enotes
\bar
\notes\ibu0f0\qa c\qa j\zq e\qa g\qa c\enotes
\endextract
\end{music}

es decir, la voz 1 (recordemos que para facilitarme el trabajo ahora las voces se numeran de abajo a arriba), luego 
la voz 4, posteriormente la 2 y la 3 a la vez y, por ultimo, la 1 otra vez.\\
Esta forma de ver las cosas es util porque el editor de patrons ritmicos esta implementado para que sea algo parecido
a esto aunque el tipo de haskell hay que completarlo con mas informacion. Por ejemplo, el ultimo patron ritmico en 
Haskell se escribiria asi:

\begin{verbatim}
patronEj1 :: PatronRitmico
patronEj1 = ( 4, [( [(1,100,False)] , 1%4),( [(4,100,False)] , 1%4),( [(2,100,False),(3,100,False)] , 1%4),( [(1,100,False)] , 1%4)] )
\end{verbatim}

\section{Encaje del Patron Ritmico}
Ahora vamos a ver como fusionar un patron ritmico con un acorde ordenado para que se forma la salida definitiva
de esta etapa, que es un tipo Music de Haskore, que ya representa musica e incluso se puede pasar a midi.

\subsection{Encaje perfecto}
Primeramente vamos a ver como encajar un patron cuando el numero de voces, tanto del patron como del acorde ordenado, 
es el mismo y cuando la duracion de ambos tambien es la misma. De esa forma todo encaja perfectamente.\\
El encaje se hace en dos partes. En una primera parte, la que corresponde mas claramente con un encaje, se sustituye
cada emph{Voz} dentro del patron ritmico por un emph{Pitch}.

\begin{verbatim}
encaja :: [Pitch] -> URV -> [(Pitch, Ligado, Acento)]
encaja lp [] = []
encaja lp ( (voz, acento, ligado) : resto) = ( lp !! (voz-1), ligado, acento) : encaja lp resto
\end{verbatim}

Posteriormente se introduce en el Pitch la duracion y el Acento formando una emph{Note} de haskore.

\begin{verbatim}
insertaDur :: Dur -> [(Pitch, Ligado, Acento)] -> [(Music, Ligado)]
insertaDur dur [] = [(Rest dur, False)]
insertaDur dur lp = insertaDur2 dur lp

insertaDur2 :: Dur -> [(Pitch, Ligado, Acento)] -> [(Music, Ligado)]
insertaDur2 dur [(pitch, ligado, acento)] = [ ( Note pitch dur [Volume acento] , ligado ) ]
insertaDur2 dur ((pitch, ligado, acento) : resto) = 
      ( Note pitch dur [Volume acento] , ligado ) : insertaDur2 dur resto
\end{verbatim}

De esa forma obtenemos una estructura que hemos llamado emph{NotasLigadas}.

\begin{verbatim}
type NotasLigadasVertical = [(Music,Ligado)] 
type NotasLigadas = [(NotasLigadasVertical,Dur)]
-- Donde Dur es la duracion del acorde
\end{verbatim}

La razon que fundamenta el uso de esta estructura es, simplemente, que Haskore no nos
da la opcion, dentro de sus constructoras del tipo Music, de hacer que una Note se ligue
con la siguiente Note. Es por eso que esto es un paso previo antes del tipo Music que
se espera que se devuelva en esta etapa y puede ser perfectamente transparente a la
persona que use este modulo.\\
EL algoritmo para resolver lo anterior es dificil de implementar aunque facil de entender.
Dada la lista que hemos llamado NotasLigadas se comienza por el final de dicha lista y se 
avanza hacia atras. Si la nota actual lleva en la tupla un True en el tipo Ligado entonces 
se ligara (si se puede) con una de las notas siguientes, que, por empezar desde la cola de 
la lista, ya estara bien ligada (nosotros la hemos llamado arreglada).\\

\begin{verbatim}
eliminaLigaduras :: NotasLigadas -> NotasLigadas
eliminaLigaduras [n] = [n]
eliminaLigaduras ((notasVerticales, dur) : resto) = 
    let { eliminadas = eliminaLigaduras resto ;
          cabeza = head eliminadas ;
          arregladoCabeza = buscaTodasNotas notasVerticales (fst cabeza)
        }  in  (fst arregladoCabeza, dur) : (snd arregladoCabeza, snd cabeza) : tail eliminadas

buscaTodasNotas :: NotasLigadasVertical -> NotasLigadasVertical -> ( NotasLigadasVertical , NotasLigadasVertical )
buscaTodasNotas notas1 notas2 = buscaTodasNotas2 notas1 [] notas2

buscaTodasNotas2 :: NotasLigadasVertical -> NotasLigadasVertical -> NotasLigadasVertical -> (NotasLigadasVertical, NotasLigadasVertical)
buscaTodasNotas2 [] notas1 notas2 = ( notas1, notas2 )
buscaTodasNotas2 (( nota, False ) : restoPitch ) notas1 notas2  = buscaTodasNotas2 restoPitch ((nota,False):notas1) notas2
buscaTodasNotas2 (( Note pitch dur lA, True ) : restoPitch ) notas1 notas2 = 
	buscaTodasNotas2 restoPitch ((Note pitch (dur + buscaNota pitch notas2) lA, False):notas1) (eliminaNota pitch notas2)
\end{vebatim}

Una vez que hemos arreglado todas las notas respecto a sus ligaduras solamente nos queda
pasar la matrix NotasLigadas a un tipo Music secuenciando las columnas y paralelizando
las filas.

\begin{verbatim}
-- deNotasLigadasAMusica: dada la lista de notas ligadas (ya sea bien arregladas o no) las pasa a musica Haskore
deNotasLigadasAMusica :: NotasLigadas -> Music
deNotasLigadasAMusica = deNotasLigadasAMusica2 (0%1)

-- deNotasLigadasAMusica2: es la funcion recursiva de deNotasLigadasAMusica y con acumulador.
-- El parametro dur indica la duracion que hay que dejar hasta el comiento de la cancion antes de interpretar 
-- las notas ligadas a tratar
deNotasLigadasAMusica2 :: Dur -> NotasLigadas -> Music
deNotasLigadasAMusica2 dur [(nV,_)] = Rest dur :+: paraleliza nV
deNotasLigadasAMusica2 dur ((nV,dur2):resto) = (Rest dur :+: paraleliza nV) :=: deNotasLigadasAMusica2 (dur + dur2) resto 


-- paraleliza: ejecuta en paralelo a lista de musica sin intereserse por el parametro booleano
paraleliza :: [( Music, Bool )] -> Music
paraleliza [] = Rest (0%1)
paraleliza [ ( nota, _ ) ] = nota
paraleliza (( nota, _ ):resto) = nota :=: paraleliza resto

\end{verbatim}

Como Haskore no posee un Music vacio usamos Rest (0%1) para referirnos a algo inutil 
musicalmente hablando.

\subsection{Problemas de encaje}
En la seccion anterior hemos hablado de como es el encaje del patron cuando el numero
de voces y la duracion de tanto el patron como del acorde ordenado eran los mismos.
Pero en la mayoria de los casos no va a ser asi. En la mayoria de los casos el 
acorde ordenado tenga diferente duracion a la del patron y muy posiblemente el numero de 
voces de ambos sera diferente. Seria un error crear un nuevo patron para cada acorde
con sus peculariedades asi que hemos introducido opciones para solucionar eso.

\subsubsection{Problemas de encaje horizontal}
Se produce cuando la duracion del patron es diferente a la duracion del acorde ordenado
al que se quiere encajar.\\
Si la duracion del patron es menor entonces el problema es menor ya que vamos a repetir
tantas veces como haga falta hasta rellenarlo completamente o hasta que nos sobre algo, 
en cuyo caso pasaremos al caso siguiente.\\
Si la duracion del patron es mayor que el del acorde o, dicho de otra forma, sobra
algo del patron cuando lo encajamos en el acorde ordenado entonces la pregunta es
?' que hacemos con dicho trozo que sobra ? A nosotros se nos ha ocurrido dos cosas:
La primera es que ese trozo se pierda comenzando el siguiente acorde con, otra vez, el 
principio del patron ritmico. La segunda es que dicho trozo sea el comienzo del siguiente
acorde, es decir, que se le pase al siguiente acorde. La primera opcion la hemos llamado
encaje horizontal no ciclico y a la segunda encaje horizontal ciclico.\\
Veamos un ejemplo para asentar ideas:\\
Sea el patron ritmico del arpegio de cuatro voces como hemos visto antes

\begin{verbatim}
- - - x
- - x -
- x - -
x - - -
\end{verbatim}
Ahora veamos el caso no ciclico
\begin{verbatim}
 - - - x - - - - - - - - - x - - - - - - x - - - x - - 
 - - x - - - x - - x - - x - - - - - - x - - - x - - - 
 - x - - - x - - x - - x - - - - x - x - - - x - - - x 
 x - - - x - - x - - x - - - x x - x - - - x - - - x - 
|-------------|-----|---------|---|-------------------|   duracion acorde
\end{verbatim}
Recordemos que la misma voz en diferentes acordes no tiene porque corresponder a la
misma altura de nota.\\
Ahora veamos el caso ciclico para la misma sucesion de acordes
\begin{verbatim}
 - - - x - - - x - - - x - - - x - - - x - - - x - - - 
 - - x - - - x - - - x - - - x - - - x - - - x - - - x 
 - x - - - x - - - x - - - x - - - x - - - x - - - x - 
 x - - - x - - - x - - - x - - - x - - - x - - - x - - 
|-------------|-----|---------|---|-------------------|   duracion acorde
\end{verbatim}

\subsubsection{problemas de encaje vertical}
Algo parecido ocurre cuando el numero de voces del acorde y el del patron es diferente pero
en este caso tenemos mas opciones. Veamos los diferentes casos.\\
Si la altura del acorde es mayor que la del patron:\\
1. Truncar: las voces extra del acorde se ignoran
\begin{verbatim}
         -
 - - - - |
 - - - - |
 - - - x | altura del acorde
 - - x - |
 - x - - |
 x - - - |
         -
\end{verbatim}
2. Saturar: supongamos que el acorde tiene A voces y el patron tiene P. En este caso
las voces desde P+1 hasta A se tocan a la vez cuando el patron toca su voz P-esima.
\begin{verbatim}
         -
 - - - x |
 - - - x |
 - - - x | <- P-esima voz del patron 
 - - x - |
 - x - - |
 x - - - | altura del acorde
         -
\end{verbatim}

Si la altura del acorde es menor que la del patron:\\
1. Truncar (no cunfundir con el truncar anterior ya que son casos distintos aunque
tengan el mismo nombre): Las voces desde A+1 hasta P no se tocan.
\begin{verbatim}
          -
 - x - -  |
 x - - -  | altura del acorde (solo 2 voces)
          -
\end{verbatim}
2. Saturar (no cunfundir con el saturar anterior ya que son casos distintos aunque
tengan el mismo nombre): Las voces desde A+1 hasta P no se tocan se toman como la voz
A-esima del acorde.
\begin{verbatim}
          -
 - x x x  |
 x - - -  | altura del acorde (solo 2 voces)
          -
\end{verbatim}
3. Ciclico: En este caso lo que se cambia es el acorde para que tenga el mismo numero
de voces que el patron. Ello se consigue repitiendo las notas del acorde pero aumentando
la octava de forma que el acorde siga ordenado en altura.
\begin{verbatim}
                                              -
 - - - x                             - - - x  |
 - - x -  -     pasaria a ser asi    - - x -  |
 - x - -  |                          - x - -  |
 x - - -  |                          x - - -  | 
          -                                   -
\end{verbatim}
4. Modulo : Si una voz X del patron esta entre el rango de A+1 y P, es decir, que cae
fuera del rango del patron, entonces esa voz se transforma en X := ((X-1) mod A) + 1. Eso
nos asegura que X cae entre 1 y A.
A-esima del acorde.
\begin{verbatim}
          -
 - x - x  |
 x - x -  | altura del acorde (solo 2 voces)
          -
\end{verbatim}

Los detalles de implementacion no merecen la pena ponerlos aqui ya que son muy simples. 
Cabe se~nalar que simplemente hay que cambiar lo que nosotros hemos llamado unidad
de ritmo vertical (\emph{URV=[(Voz,Acento,Ligado)]}) a~nadiendo o quitando mas voces
para que se ajusta a las voces del acorde.

\end{document}

% hablar de patrones ritmicos dinamicos, que se generen en tiempo de llamada, como el arpegio a n voces











